\chapter{Open Issues}
\label{chapter:openissues}

In this chapter we list some open issues and improvements.


\section{Locking Transformation Engine}
The current implementation of the application layer uses the write lock
provided by the \texttt{Abstract\-Document} of the Swing 
text package to implement the session lock. 
Although this works flawlessly, it has some performance
implications. Receiving requests from the network layer passes through
the following steps:

\begin{enumerate}
 \item aquire the session lock
 \item transform the request
 \item pass the transformed operation to the application layer, which applies
       the operation to the local document
 \item release the session lock
\end{enumerate}

Swing's \texttt{Abstract\-Document} has a read/write lock. The read lock
is used for instance while rendering the document. The write lock is used
only by \texttt{Abstract\-Document} as well as its subclasses whenever the
document is modified (for instance when calling the \texttt{insert\-String}
method).

The transformation process requires that from the point where the transformation
starts to the point where the operation is inserted, the document is not 
modified. While executing step 2, the transformation process does only
need a read lock. However step 3 requires a write lock on the document.
Unfortunately, it is not possible to upgrade from a read lock to a write
lock. So the only available option is to aquire a write lock from the
beginning.

However, this has the disadvantage that while transforming
the request (which could take some significant time), the AWT thread is
not allowed to render the document. Rendering the document is a 
frequently executed operation. For example whenever the caret blinks, a
repaint has to occur to repaint the damaged region.

It is not possible to simply subclass a \texttt{Document}
implementation and overwrite the lock related methods because these
methods (\texttt{read\-Lock}, \texttt{read\-Unlock}, \texttt{write\-Lock},
and \texttt{write\-Unlock}) are all declared final.


\subsection{Solution}
The only viable solution seems to be to completely reimplement the 
\texttt{Document} interface. But the problem is not solved alone with
reimplementing the \texttt{Document} interface. In lots of places in
the Swing text package instanceof checks for the
\texttt{Abstract\-Document} class are made (especially to support locking and
the \texttt{replace} method). So quite a lot of other Swing classes would
have to be partially reimplemented. Definitely not a small task.

Nonetheless, the achieved performance improvement could be well worth the 
effort. Before starting this task, we should try to determine how much
the performance could be improved.



\section{Dependency of Collaboration Layer on Swing}
The collaboration layer makes use of an \texttt{AbstractDocument} subclass to
keep the server copy of the document up-to-date, which introduces a
dependency of the collaboration
layer dependent on Swing. This may not seem a big issue. However, if 
somebody would want to write an application layer based on SWT (windowing
toolkit used by Eclipse), for instance an Eclipse plugin, this would
cause problems. On some platforms it is not possible to have both Swing (AWT)
and SWT alongside in the same application. Therefore it is not desirable
to have a dependency on Swing in the collaboration layer, because it limits
the applicability of it.


\subsection{Solution}
Implementing a custom document model for the server copy of the document
would be a fairly involved task. An efficient implementation of the
document storage would have to be implemented, similar to the implementation
of the \texttt{Content} interface of the Swing package. The implementation
must also support keeping track of the document structure. 
This implementation effort would be more than compensated by the 
increased applicability of the collaboration layer.



\section{Crashes Caused by Bonjour for Java}
There occured an undeterminstic JVM behavior in combination with the Bonjour framework. Sometimes the JVM crashed at startup time due to some Bonjour framework exception. The exception might be caused by the mDSNResponder daemon. This behavior was mostly observed on the Linux OS. 

The problem will have to be communicated in detail to the Bonjour developer community.

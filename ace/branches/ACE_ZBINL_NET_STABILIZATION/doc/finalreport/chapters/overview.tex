\chapter{Overview}
\label{chapter:overview}



\section{What is Collaborative Editing?}
Collaborative editing happens when several users edit a document at the same time. The users may be geographically dispersed but are connected by communication networks. Every change made by one user is shown immediately to all other users. Thus all users observe the same document content. All participants of an editing session are allowed to concurrently and independently edit any part of the document at any time. Thus a free and natural information flow is guaranteed. 

For example, instead of passing a document from one team member to another for review, all of  them could review it simultaneously through collaborative editing.



\section{Use Cases for Collaborative Editing}
Following is a list with possible use cases for collaborative editing.

\begin{itemize}
\item Five correctors from a publishing company correct simultaneously a novel to be released. They all use their own computer but are connected using a collaborative editing system.
\item The management of a global technology company is doing a brainstorming simulatenously. Each member, geographically dispersed from the others, writes his ideas on the shared document.
\item Extreme Programming (XP): An XP practice is to have two team members that collaboratively write software. That is, one writes the code and the other one observes it. Instead of both sitting in front of the same computer, they could use a collaborative editing system that would enable them to work with their own computer. They could also program together at the same time.
\item A lecturer uses a collaborative editing system to demonstrate programming techniques to his students and help students in solving exercises (collaborative solving). Each student as well as the lecturer has an own computer.
\item A group writes a report together. Instead of passing the document from member to member, all work with a collaborative editing system together at the same time in one document and thus save a lot of time.
\end{itemize}



\section{Requirements of a Collaborative Application}
A synchronous collaborative editor has some unique requirements.
The following requirements have been identified for such an editor
by Sun et al. (\cite{sun98a}).

\subsection{Real-time} 
The response to local user actions must be quick, ideally
as quick as with a single-user editor. Imposing a global total order on
message transmission is not
an option because of the distributed nature of the system. Thus, generally 
these systems replicate the document at each user's site.

\subsection{Distributed} 
Cooperating users reside on different machines 
connected by communication networks with nondeterministic latency.

\subsection{Unconstrained} 
Multiple users are allowed to concurrently and
independently edit any part of the document at any time, in order to 
facilitate free and natural information flow among multiple users.



\section{Previous Work}
\label{sect:overview.previouswork}

The diploma project is based on the work done in the semester project. Some
additional work has been done from the end of the semester
project to the start of the diploma project. In this section we show which
results are not part of the diploma project, but have been implemented
before.

\subsection{Semester Project}
In the semester project we have implemented the following features:
\begin{itemize}
 \item consistency control algorithm (\emph{Jupiter})
 \item test framework for algorithm
\end{itemize}

In the first stage of the semester project we evaluated different
consistency control algorithms. The \emph{Report Evaluation Algorithm}
documents the evaluation process. The implementation of the
algorithm is documented in \emph{Report Implementation Algorithm}. The
test framework is documented in \emph{Report Testframework}.

Further, we did some research to find out how we want to implement the
network as well as the application layer. The results of this research
can be found in the \emph{Report Evaluation Network} and 
\emph{Report Evaluation GUI} respectively.


\subsection{Summer Time}
During the summer break we tried to finish the implementation of undo/redo
in the algorithm. However, we stumbled across some issues that could not
be resolved in reasonable time. Read section \ref{sect:algorithm.undoredo} for
a detailed description of the problem.

We also investigated how difficult it would be to create a collaborative 
editing plugin for jEdit (see \href{http://jedit.org/}{http://jedit.org/}).
More about this investigation can be read in chapter 
\ref{chapter:decisionsapplication}.

The test framework of the algorithm has also been extended. The following 
new features have been added:
\begin{itemize}
 \item support to specify scenarios as used by \emph{Jupiter} (client-server model)
 \item support to specify scenarios with undo/redo events (unfortunately, undo/redo is not used by our algorithm).
\end{itemize}

Further, we tried to implement the \emph{adOPTed} algorithm (see 
\cite{ressel96}) with the scripting language \emph{Ruby} (see
\href{http://www.ruby-lang.org/en/}{http://www.ruby-lang.org/en/}). The
implementation of the basic algorithm succeeded. However, the description
of undo/redo proved to be too inaccurate to implement. We therefore 
decided to stick with \emph{Jupiter}, as using \emph{adOPTed} posed too many
risks.

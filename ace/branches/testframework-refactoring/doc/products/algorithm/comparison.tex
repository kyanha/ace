\section{Comparison of Algorithms}

\subsection{Overview}

The following tables compare different aspects of the examined algorithms. The first table explains what is meant by the aspects.

\newcommand{\acol}[1]{\multicolumn{1}{|p{1.6in}|}{\small{#1}}}
\newcommand{\dcol}[1]{\multicolumn{1}{|p{3.9in}|}{\small{#1}}}

 \begin{table}[!ht]
  \begin{tabular}{|l|c|}
   \hline
    \headercol{1.6in}{Aspect} & 
    \headercol{3.9in}{Description}  \\
   \hline
    \acol{Year} & 
    \dcol{The year the research paper was published.} \\
   \hline 
    \acol{Correct?} & 
    \dcol{No if another research paper has proved the algorithm (or its OT functions) wrong, otherwise Yes} \\
   \hline 
    \acol{Architecture} & 
    \dcol{Which architecture is the algorithm designed for?} \\
   \hline 
    \acol{Available Information} & 
    \dcol{Is there enough information for an implementation?} \\
   \hline 
    \acol{Intention Preservation} & 
    \dcol{How is the user's intention for an operation be preserved? (see \ref{constraints})} \\
   \hline 
    \acol{Causality Preservation} & 
    \dcol{How is causal ordering relation on operations preserved? (see \ref{constraints})} \\
   \hline 
    \acol{Copies Convergence} & 
    \dcol{How is copy convergence on all replicated objects achieved? (see \ref{constraints})} \\
   \hline 
    \acol{Broadcast} & 
    \dcol{When is the broadcast i.e. the emission of a generated operation carried out?} \\
   \hline 
    \acol{Delivery} & 
    \dcol{In what order are the operations at each site executed?} \\
   \hline 
    \acol{Undo} & 
    \dcol{Is a user undo functionality for the algorithm available?} \\
   \hline
  \end{tabular}
  \caption{Aspect Explanation}
 \end{table}

\newcommand{\ccol}[1]{\multicolumn{1}{|p{0.8in}|}{\tiny{#1}}}

\begin{table}[H]
 \begin{tabular}{|l|c|c|c|c|c|}
  \hline
   \headercol{0.8in}{} &
   \headercol{0.8in}{dOPT} &
   \headercol{0.8in}{Jupiter} &
   \headercol{0.8in}{adOPTed} &
   \headercol{0.8in}{GOT} &
   \headercol{0.8in}{GOTO} \\
  \hline
  \hline
   \ccol{Year} &
   \ccol{1989} &
   \ccol{1995} &
   \ccol{1996} &
   \ccol{1998} &
   \ccol{1998} \\
  \hline
   \ccol{Correct?} &
   \ccol{no} &
   \ccol{control algorithm} &
   \ccol{control algorithm} &
   \ccol{yes} &
   \ccol{control algorithm} \\
  \hline
   \ccol{Architecture} &
   \ccol{replicated, multicast} &
   \ccol{replicated, unicast} &
   \ccol{replicated, multicast} & 
   \ccol{replicated, multicast} &
   \ccol{replicated, multicast} \\
  \hline
   \ccol{Available Information} &
   \ccol{enough} &
   \ccol{enough} &
   \ccol{enough} & 
   \ccol{enough} &
   \ccol{enough} \\
  \hline
  \hline
   \ccol{Intention Preservation} &
   \ccol{dOP Transformation} &
   \ccol{Transformation and two-dimensional graph} &
   \ccol{L-Transformation and multidimensional graph} &
   \ccol{IT and ET} &
   \ccol{IT and ET} \\
  \hline 
   \ccol{Causality Preservation} &
   \ccol{state vectors} &
   \ccol{state vectors} &
   \ccol{state vectors} &
   \ccol{state vectors} &
   \ccol{state vectors} \\
  \hline
   \ccol{Copies Convergence} &
   \ccol{TP1 (but no convergence achieved)} &
   \ccol{TP1} &
   \ccol{TP1 and TP2} &
   \ccol{non-continuous global order and undo/redo} &
   \ccol{TP1 and TP2} \\
  \hline
  \hline
    \ccol{Broadcast} &
    \ccol{immediate} &
    \ccol{immediate} &
    \ccol{immediate} &
    \ccol{immediate} &
    \ccol{immediate} \\
  \hline
   \ccol{Delivery} &
   \ccol{causal order} &
   \ccol{causal order} &
   \ccol{causal order} &
   \ccol{causal order} &
   \ccol{causal order} \\
  \hline
  \hline
   \ccol{Undo} &
   \ccol{no} &
   \ccol{no (but could be derived from adOPTed)} &
   \ccol{yes} &
   \ccol{no} &
   \ccol{yes} \\
  \hline
 \end{tabular}
 \caption{Comparison Matrix 1}
\end{table}

\begin{table}[H]
 \begin{tabular}{|l|c|c|c|c|c|}
  \hline
   \headercol{0.8in}{} &
   \headercol{0.8in}{SOCT2} &
   \headercol{0.8in}{SOCT3} &
   \headercol{0.8in}{SOCT4} &
   \headercol{0.8in}{SDT} &
   \headercol{0.8in}{TIBOT} \\
  \hline
  \hline
   \ccol{Year} &
   \ccol{1997} &
   \ccol{2000} &
   \ccol{2000} &
   \ccol{2004} &
   \ccol{2004} \\
  \hline
   \ccol{Correct?} &
   \ccol{no (transformation functions)} &
   \ccol{yes} &
   \ccol{yes} &
   \ccol{no (transformation functions)} &
   \ccol{yes} \\
  \hline
   \ccol{Architecture} &
   \ccol{replicated, multicast} &
   \ccol{replicated, multicast} &
   \ccol{replicated, multicast} & 
   \ccol{replicated, multicast} &
   \ccol{replicated, multicast} \\
  \hline
   \ccol{Available Information} &
   \ccol{enough} &
   \ccol{implementation of sequencers} &
   \ccol{implementation of sequencers} & 
   \ccol{not enough for implementation} &
   \ccol{not enough} \\
  \hline
  \hline
   \ccol{Intention Preservation} &
   \ccol{IT and ET} &
   \ccol{IT and ET} &
   \ccol{only IT} &
   \ccol{IT and ET} &
   \ccol{IT} \\
  \hline 
   \ccol{Causality Preservation} &
   \ccol{state vectors} &
   \ccol{timestamps} &
   \ccol{timestamps} &
   \ccol{state vectors} &
   \ccol{time intervals} \\
  \hline
   \ccol{Copies Convergence} &
   \ccol{TP1 and TP2} &
   \ccol{TP1 and continous global order} &
   \ccol{TP1 and continous global order} &
   \ccol{TP1 and TP2} &
   \ccol{TP1, propagation and synchronization rules} \\
  \hline
  \hline
    \ccol{Broadcast} &
    \ccol{immediate} &
    \ccol{immediate (as soon as timestamp is assigned)} &
    \ccol{deferred, in timestamp order} &
    \ccol{immediate} &
    \ccol{deferred, after time interval} \\
  \hline
   \ccol{Delivery} &
   \ccol{causal order} &
   \ccol{continous global order} &
   \ccol{continous global order} &
   \ccol{causal order} &
   \ccol{?\footnotemark} \\
  \hline
  \hline
   \ccol{Undo} &
   \ccol{no} &
   \ccol{no} &
   \ccol{no} &
   \ccol{no} &
   \ccol{no} \\
  \hline
 \end{tabular}
 \caption{Comparison Matrix 2}
\end{table}
\footnotetext{Due to lack of information we are not capable of determining the answer definitely.}

\begin{table}[H]
 \begin{tabular}{|l|c|c|c|c|c|}
  \hline
   \headercol{0.8in}{} &
   \headercol{0.8in}{NICE} &
   \headercol{0.8in}{LI04} &
   \headercol{0.8in}{CCU} &
   \headercol{0.8in}{} &
   \headercol{0.8in}{} \\
  \hline
  \hline
   \ccol{Year} &
   \ccol{2002} &
   \ccol{2004} &
   \ccol{1995} &
   \ccol{} &
   \ccol{} \\
  \hline
   \ccol{Correct?} &
   \ccol{yes} &
   \ccol{yes} &
   \ccol{probably yes} &
   \ccol{} &
   \ccol{} \\
  \hline
   \ccol{Architecture} &
   \ccol{replicated, unicast} &
   \ccol{replicated, multicast} &
   \ccol{replicated, unicast} & 
   \ccol{} &
   \ccol{} \\
  \hline
   \ccol{Available Information} &
   \ccol{enough} &
   \ccol{not enough \footnotemark} &
   \ccol{not enough} & 
   \ccol{} &
   \ccol{} \\
  \hline
  \hline
   \ccol{Intention Preservation} &
   \ccol{IT} &
   \ccol{IT and ET} &
   \ccol{?} &
   \ccol{} &
   \ccol{} \\
  \hline 
   \ccol{Causality Preservation} &
   \ccol{central notification server} &
   \ccol{state vectors} &
   \ccol{?} &
   \ccol{} &
   \ccol{} \\
  \hline
   \ccol{Copies Convergence} &
   \ccol{TP1 and unique global order} &
   \ccol{TP1 and TP2} &
   \ccol{TP1 and TP2} &
   \ccol{} &
   \ccol{} \\
  \hline
  \hline
    \ccol{Broadcast} &
    \ccol{immediate} &
    \ccol{immediate} &
    \ccol{immediate} &
    \ccol{} &
    \ccol{} \\
  \hline
   \ccol{Delivery} &
   \ccol{causal order} &
   \ccol{causal order} &
   \ccol{causal order} &
   \ccol{} &
   \ccol{} \\
  \hline
  \hline
   \ccol{Undo} &
   \ccol{no} &
   \ccol{no} &
   \ccol{no} &
   \ccol{} &
   \ccol{} \\
  \hline
 \end{tabular}
 \caption{Comparison Matrix 3}
\end{table}
\footnotetext{The paper \cite{li04a} with the necessary implementation details should be released around May 2005.}

\subsection{Selection Criteria}

From the set of available algorithms we want to make a pre-selection. This pre-selection is based on the criteria set forth in this section.

\paragraph{Correctness:} This is obviously the most important criteria. If an algorithm is not correct, it is not worth being implemented.

\paragraph{Availability of information:} Some papers do not provide enough information for an implementation.

\paragraph{Availability of user undo:} Users of collaborative applications expect the same commands as in a single user application. Without user level undo, the user experience will not be satisfactory.

\paragraph{Algorithmic complexity:} Some algorithms are strikingly simple, others are very complex (too complex). Simplicity is a selection criteria.


\subsection{Selection of Algorithms}

\subsubsection{Correctness}
Based on the first selection criteria (correctness) \emph{dOPT} algorithm is deemed unsuitable. Further the transformation functions used by some algorithms have been proved incorrect \cite{imine04}. These include \emph{adOPTed}, \emph{GOTO}, \emph{SOCT2} and \emph{SDT}. Note that the transformation functions can be replaced as they are independent of the control algorithm. By using the proposed IT function of \cite{imine04} these control algorithms would be correct again (at least the control algorithm of \emph{adOPTed} and \emph{GOTO} have been proved correct). As \emph{GOTO}, \emph{SOCT2} and \emph{SDT} also use ET functions and \cite{imine04} only proposed and proved IT, we cannot assume them to be correct for an implementation yet. There have to be ET functions developed and proved analogous to the IT functions in \cite{imine04} first.

The following algorithms meet the correctness criteria:
\begin{itemize}
 \item \emph{Jupiter} (see \ref{algo:jupiter})
 \item \emph{adOPTed} (see \ref{algo:adopted})
 \item \emph{GOT} (see \ref{algo:got})
 \item \emph{SOCT 3/4} (see \ref{algo:soct3} and \ref{algo:soct4})
 \item \emph{TIBOT} (see \ref{algo:tibot})
 \item \emph{NICE} (see \ref{algo:nice})
 \item \emph{LI04} (see \ref{algo:li04})
 \item \emph{CCU} (see \ref{algo:ccu})
\end{itemize}

It is important to note that \emph{CCU}, \emph{TIBOT} and \emph{LI04} have not been approved or successfully implemented by a third party as far as we know.


\subsubsection{Availability of Information}
For the algorithms in the next list we do not have enough information available for an implementation.

\begin{itemize}
 \item \emph{CCU} (see \ref{algo:ccu})
 \item \emph{SDT} (see \ref{algo:sdt})
 \item \emph{TIBOT} (see \ref{algo:tibot})
\end{itemize}

For the implementation of \emph{SOCT3} and \emph{SOCT4} some more papers are needed concerning the implementation of sequencers. Though referenced, these could not be found on the Internet. 

For \emph{Li04} we do not have enough information yet. But there will be a paper with implementation details published around May 2005 according to Mr. Du Li.


\subsubsection{Complexity}
\emph{SOCT2} is considered to be very complex \cite{imine03b} \cite{sdt} due to management of the sets $s_i$ and $b_i$ associated with each $Insert$ operation. \emph{SOCT3} and \emph{SOCT4} use a sequencer that simplifies the implementation of the algorithm but requires this additional component which complicates the implementation. The theoretical aspects of \emph{GOT} as well as \emph{GOTO} are relatively complex, but the implementation itself should be straightforward. \emph{NICE} uses a central notification server that simplifies the algorithm significantly. \emph{Jupiter} is a very simple algorithm, because it restricts itself to point-to-point communication. \emph{adOPTed} is more complicated than \emph{Jupiter} because it adds $n$-way communication. However, we consider the implementation as relatively straightforward.  

The following algorithms are considered to be too complex to be accounted for an implementation.
\begin{itemize}
 \item \emph{SOCT2} (see \ref{algo:soct2})
\end{itemize}


\subsubsection{User Undo}
The following algorithms have a user level undo defined. This does not imply that it would be impossible to devise an undo mechanism for other algorithms. Note however that adding an undo mechanism to an existing algorithm is a non-trivial task.

\begin{itemize}
 \item \emph{Jupiter} (see \ref{algo:jupiter}) 
 \item \emph{adOPTed} (see \ref{algo:adopted})
 \item \emph{GOTO} (see \ref{algo:goto})
\end{itemize}


\subsubsection{Selected Algorithms}
Based on the above observations but excluding the user undo requirement, the following algorithms remain as implementation choice:

\begin{itemize}
 \item \emph{Jupiter} (see \ref{algo:jupiter})
 \item \emph{adOPTed} (see \ref{algo:adopted})
 \item \emph{GOT} (see \ref{algo:got})
 \item \emph{NICE} (see \ref{algo:nice})
\end{itemize}

If we take the user undo functionality into account, the selection shrinks to three algorithms:
\begin{itemize}
 \item \emph{Jupiter} (see \ref{algo:jupiter}) 
 \item \emph{adOPTed} (see \ref{algo:adopted})
\end{itemize}

All the above algorithms fulfill the selection criteria sufficently well.

\section{Pers�nliche Bilanzen}
\subsection{Mark Bigler}
Ich bin sehr froh dar�ber, dass unser Vorschlag f�r das Semesterprojekt angenommen wurde. Dadurch konnten wir auf einem Gebiet arbeiten, welches jeden von uns interessiert hatte. Zus�tzlich haben alle von uns mehr Schulstunden im Wintersemester als im Sommersemester ausgew�hlt, was uns die M�glichkeit gab, die Zeit f�r Arbeiten am Projekt relativ flexibel zu gestallten. Im Grossen und Ganzen war es eine sehr gute Erfahrung. Zum Beispiel hatte ich w�hrend der Implementation des Algorithmus zum ersten Mal die M�glichkeit gelernte Software-Patterns aus dem zweiten Studienjahr anzuwenden.

\subsubsection{Positive Punkte}
\begin{itemize}
 \item Sehr gut harmonierends und motiviertes Team.
 \item Wir haben uns ein interessantes Thema ausgew�hlt und waren somit sehr motiviert.
 \item Dank dem Zeitplan konnten wir die Arbeit gut ein- und aufteilen.
 \item Wir konnte uns realistische Ziele setzen und haben diese auch alle erreicht.
\end{itemize}

\subsubsection{Negative Punkte}
Sehr viel Aufwand f�r Projekt Management Dokumente. Wahrscheinlich h�tten wir sehr gut weniger machen k�nnen, ohne dass damit Konsequenzen f�r den Verlauf oder die Qualit�t des Projektes entstanden w�ren.

\subsubsection{Selbstkritik}
Ich habe die Zeit w�hrend den Fr�hlingsferien nicht optimal ausgenutzt. Deshalb konnte ich den GUI Bericht auch nicht bis zum geplanten Termin fertigstellen. Dies hatte aber keinen Einfluss auf weiteren Aktivit�ten des Projektes.

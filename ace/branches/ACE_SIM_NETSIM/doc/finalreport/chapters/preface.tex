\chapter{Preface}
\label{chapter:preface}

This report contains the documentation
of the ACE diploma project and is split into several parts. Each part has
several chapters. The following lists all parts and their chapters and
describes what can be found where.



\section*{Part 1: Introduction}
The first part of the book gives a general introduction to the diploma
project as well as to the concepts on which ACE is built.

\begin{description}
 \item[Chapter 1: Overview] This chapter describes first what collaborative editing is and it describes some common use cases of collaborative editing. Further it lists the requirements for a real-time collaborative application. In the last section of the overview the work we did prior to the diploma project is listed.
 \item[Chapter 2: Functionality] The second chapter describes the functionality of the application. The terminology used throughout this report is described.
 \item[Chapter 3: Algorithm] The last chapter in the introduction part introduces the reader to the concepts of operational transformation. It also lists the selection criteria we used to select the algorithm. Then it goes on to describe the chosen algorithm and its implementation. 
\end{description}


\section*{Part 2: Decisions and Justifications}
In the semester project we evaluated different technologies for the
implementation of the application as well as the network layer. Here we 
describe our decisions and we justify these decisions.

\begin{description}
 \item[Chapter 4: Decisions for Network Implementation] In this chapter we describe the decisions we made to implement the network layer. The decisions are justified based on a set of selection criteria.
 \item[Chapter 5: Decisions for Application Implementation] Similarly to the last chapter, we describe here which decisions we made before implementing the application layer and justify these decisions.
\end{description}

 
\section*{Part 3: Implemented Solution}
The third part contains a detailed description of the implemented solution.
These descriptions are separated into a general chapter as well as a
chapter for each layer.

\begin{description}
 \item[Chapter 6: Architectural Overview] The fist chapter of this part describes the general architecture of ACE as well as the interfaces between the three layers of ACE.
 \item[Chapter 7: Application Layer] This chapter documents the application layer implementation of ACE.
 \item[Chapter 8: Collaboration Layer] This chapter describes the implementation of the collaboration layer.
 \item[Chapter 9: Network Layer] The final chapter of this part documents the implementation of the network layer.
\end{description}

 
\section*{Part 4: Analysis}
The fourth part of this document analyzes the diploma project. First, we
show which goals have been achieved and which not. Second, we highlight
the differences between the planned amount of time and the actual used
amount of time for each work packages. Further, the most important open
issues are listed and we describe what we have learned during the project.
The last two chapters of this section are dedicated to make an outlook
of potential additional features and a conclusion.

\begin{description}
 \item[Chapter 10: Achievement of the Goals] In this chapter we describe which goals we have achieved and which we have not.
 \item[Chapter 11: Differences to Project Plan] Here, the differences between the project plan and the reality are highlighted and discussed.
 \item[Chapter 12: Open Issues] In every project, there are always some open issues. This chapter lists the most important issues and shows how these issues could be resolved.
 \item[Chapter 13: Lessons Learned] We have learned a lot from a project like ACE. The lessons we have learned in this project are documented in this chapter.
 \item[Chapter 14: Outlook] The time available in the diploma project was short. There are still a lot of features that would be worth being implemented. Some ideas of the features we could implement in the future are described in this chapter.
 \item[Chapter 15: Conclusion] The final chapter of the part contains the conclusion of the whole project.
\end{description}
 
 
\section*{Part 5: Appendix}
The appendices contain a list of used references, frameworks and libraries as well as a list of used abbreviations and terms. Further some selected frameworks and libraries
are described.

\begin{description}
 \item[Appendix A: References] This appendix contains a list of used references. This includes technical articles, documentation from the semester project, and websites.
 \item[Appendix B: Used Technologies] The appendix B lists all the dependencies of ACE and their versions. A short introduction is given for some selected and important frameworks used by ACE.
 \item[Appendix C: Thread Domains] In this appendix we describe the concept, idea, and implementation of thread domains. Thread domains were implented as part of the diploma project and are documented here as reference for interested readers.
 \item[Appendix D: Glossary] This appendix explains the used abbreviations and terms.
 \item[Appendix E: Responsabilities] The last appendix shows who did which part of the documentation and of the source code.
\end{description}

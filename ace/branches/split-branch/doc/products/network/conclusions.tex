\section{Conclusions}
We have discussed several technologies in this report. This information will be useful when selecting a network technology for the diploma project. Before we can select a technology, we have to specify the requirements.


\subsection{Criterias}
In order to select a particular network technology or a combination of several technologies we have to answer some basic questions, both concering the architecture and some political decissions.

\paragraph{Open Architecture:} One basic question to answer is whether we want to have an open or a closed protocol. If we decide to have an open, public protocol, other applications could collaborate with our application. We would have to define the protocol and make this information publicly available. An open protocol would greatly benefit if it were not tied to any programming language, that is in our case \emph{Java}. \emph{Jini/RMI} for instance would pretty much limit the collaboration with other applications to \emph{Java} applications. A network layer based on either \emph{JXTA} or a combination of \emph{Bonjour} and a \emph{BEEP} profile would be programming language independent and thus more broadly applicable.

\paragraph{Centralization:} We have to decide whether we want to have some kind of central server (possibly several loadbalancing servers) or not. \emph{JXTA} does work without any central server, but does it work good enough (performance, latency, discovery time, ...)? A central server would mean that a central, publicly available server must be maintained by somebody. 
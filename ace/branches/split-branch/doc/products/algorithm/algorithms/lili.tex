\subsection{Li 04}
\label{algo:li04}

This paper was written 2004 by Du Li and Rui Li in \cite{li04} \footnote{Their approach was not named in \cite{li04}, so we name it \emph{Li 04} for convenience.}.  In fact, the algorithm \emph{Li 04} replaces SDT (\ref{algo:sdt}). It proposes a novel transformation based approach to preserving the correct operation effects relation in group editors. Due to its root in effects relation violation (see example in \ref{algo:sdt}), the divergence problem is then solved automatically so that convergence is achieved in the presence of arbitrary transformation paths.
In this summary, firstly two consistency model issues are introduced and secondly the control algorithm with its transformation functions are explained.

\subsubsection{Operation effects relation}
The position of every character in a string is unique. Therefore the position relation between all characters in any string is a total order $\prec$. Characters in any document state are eventually a result of editing operations. Every characterwise operation $O$ has an effect character $C(O)$. Hence the total order $\prec$ of characters is extended to operations. Given any two operations $O_{x}$ and $O_{y}$, it is $O_{x} \prec O_{y}$ iff $C(O_{x}) \prec C(O_{y})$. Relation $\prec$ is not defined between any pair of inverse operations because the effect characters of inverse operations are the same, their position relation is not a total order.

\subsubsection{Definition of the CSM consistency model}
A group editor is CSM-consistent iff the following three conditions hold:
\begin{itemize}
 \item \textbf{C}ausality preservation: all operations are executed in their cause-effect order.
 \item \textbf{S}ingle-operation effects preservation: the effect of executing any operation in any execution state achieves the same effect as in its generation state.
 \item \textbf{M}ulti-operation effects relation preservation: the effects relation of any two operations maintains after they are executed in any states.
\end{itemize}

\subsubsection{Transformation functions}
Operational transformation functions are used for achieving CSM consistency in group editors.

\paragraph{Inclusion Transformation}
The IT function is defined in the straightforward way except that it uses the operation effects relation $\prec$. If the correct relation $\prec$ between the effects of all concurrent operations is known, then remote operations can always be executed correctly while preserving S and M conditions. The problem is how to determine the effects relation. Therefore, the concept of last synchronization point (LSP) is introduced. Let $V_{1}$ and $V_{2}$ be the state vectors of operations $O_{1}$ and $O_{2}$, respectively. Let $V_{min}$ be a state vector, each element of which is equal to the minimal value of the corresponding elements of $V_{1}$ and $V_{2}$. Then $S_{lsp}$ is the state that corresponds to $V_{min}$. State $S_{lsp}$ is the last common state for two operations to be transformed inclusively. By means of the LSP the effects relation $\prec$ between two concurrent operations can be determined.

\paragraph{Exclusive Transformation}
The conceptual ET function is defined similarly to the IT function described above based on the effects relation $\prec$. The question is again how to know $\prec$. For the ET function to work correctly, position parameters are used as far as possible. Three different cases are identified to determine $\prec$ (see 4.3 in \cite{li04} for further details). The three cases reveal the following important fact: Given two operations, $O_i^{S^{i-1}}$ and $O^{S^i}$ where $O_i \rightarrow O$, and $O$ does not depend on $O_i$, only when $S^i$ is the generation state of $O$, is it safe to use position parameters to determine the effects relation, and are the original ET functions of \cite{sun98a} equivalent to the conceptual ET functions proposed. Therefore effects relation cannot always be determined correctly. This is allowed for in the control algorithm described next.

\subsubsection{The control algorithm}
The control algorithm partly follows the structure of \emph{GOTO} (see \ref{algo:goto}). A history buffer is used at each site. When a remote operation $O$ is causally-ready for execution, a copy of HB in SQ is made. SQ is transposed such that all operations in the left subsequence causally precede $O$, and all operations in the right subsequence are concurrent with $O$. After that, $O$ is inclusively transformed against the concurrent subsequence in SQ. After the result $O'$ is executed in the current state and appended to HB. The details of the proposed algorithm as well as the differences from GOTO are presented in section 5 in \cite{li04}. The algorithm makes use of concepts described above: operation effects relation and LSP to perform IT and ET correctly. The concept of state difference SD (see also algorithm \emph{SDT} in \cite{sdt}) is proposed further and used in relation to excluding operation effects (ET). SD allows for the ET problem implied above so that the effects of contextually preceding operations can be excluded correctly.

\subsubsection{Properties}
\begin{itemize}
 \item IT and ET functions are based on a novel consistency model
 \item control algorithm similar to \emph{GOTO} (see \ref{algo:goto})
 \item proves to satisfy TP1 and TP2
 \item no user undo functionality proposed
 \item architecture: replicated, multicast
\end{itemize}
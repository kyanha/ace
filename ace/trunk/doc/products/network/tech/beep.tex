\section{BEEP}
\label{sect:beep}

\emph{BEEP} stands for Blocks Extensible Exchange Protocol. It is a protocol framework for building application protocols. Many Internet protocols reinvent a set of basic functions. The most common include:

\begin{itemize}
 \item Framing: separating each request from the next
 \item Matching responses to requests
 \item Pipelining: permitting multiple outstanding requests
 \item Multiplexing: permitting multiple asynchronous requests
 \item Reporting errors
 \item Negotiating encryption
 \item Negotiating authentication
\end{itemize}

\emph{BEEP} specifies reusable tools for all these functions, instead of requiring the same decisions to be made over again for each new application. It provides a framework that integrates existing Internet standards for encryption and authentication and new standards for connection management. A networked application needs merely to supply the \emph{important} part, the part that distinguishes it from other applications, as a \emph{BEEP} profile. The mundane parts are the same for all protocols, and therefore can be coded as a library, freeing the application designer to focus on the interesting bits.

The good thing about \emph{BEEP} is that it solves common problems of most connection-oriented application protocols just once inside a framework. Further it is very efficient, requiring only a minimal overhead.

\emph{BEEP} is an inappropriate protocol if the protocol requires multicast or real-time support (for instance multicast video streaming).



\subsection{Concepts}
\emph{BEEP} is a peer-to-peer protocol in the sense that there is no notion of client or server. For convenience we'll refer to the peer that starts a connection as the \emph{initiator}, and the peer accepting the connection as the \emph{listener}. When a connection is established between the two, a \emph{BEEP} session is created.

\subsubsection{Channels}
All communication in a session happens within one or more \emph{channels}. The peers require only one IP connection, which is then multiplexed to create channels. The nature of communication possible within that channel is determined by the \emph{profiles} it supports (each channel may have one or more).

The first channel, channel 0, has a special purpose. It supports the \emph{BEEP} management profile, which is used to negotiate the setup of further channels. The supported profiles determine the precise interaction between the peers in a particular channel. Defining a protocol in \emph{BEEP} comes down to the definition of profiles.

\subsubsection{Profiles}
After the establishment of a session, the initiator asks to start a channel for the particular profile or set of profiles it whishes to use. If the listener supports the profile(s), the channel will be created. Profiles themselves take one of two forms: those for initial tuning and those for data exchange.

\emph{Tuning profiles}, set up at the start of communication, affect the rest of the session in some way. For instance, requesting the TLS profile ensures that channels are encrypted using Transport Layer Security. Other tuning profiles perform steps such as authentication.

\emph{Data exchange profiles} set expectations between the two peers as to what sort of exchanges will be allowed in a channel, similar to the was Java interfaces set expectations between interacting objects as to what methods are available. A profile is identified by a URI.

\subsubsection{Types of Data}
\emph{BEEP} puts no limits on the kind of data a channel can carry. It uses the MIME standard to support payloads of arbitrary type.



\subsection{Prototype}
\subsubsection{Echo Server}
The \emph{BEEP} implementation of an echo server can be found in \texttt{EchoServer}. The code is straighforward. First a profile registry (\texttt{ProfileRegistry}) has to be created. All the supported profiles (\texttt{Profile}) have to be added to the created profile registry by calling \texttt{addStartChannelListener}. The profiles are registered with an URI that is used as a key to specify a specific profile when creating a new channel.

Once the registry is set up and the profiles added, the server can start listening for client connections. The class \texttt{TCPSessionCreator}'s \texttt{listen} method is used for this purpose. The server does not need to do anything special with the returned session as all the communication is handled by the selected profile.

The echo server uses the existing \texttt{EchoProfile} profile. This profile implements the echo behavior.

\subsubsection{Echo Client}
The echo client (see \texttt{EchoClient}) connects to the echo server and sends echo requests using the BEEP echo profile. First a session has to be created to the server. The method \texttt{initiate} in \texttt{TCPSessionCreator} is used for that purpose. Once we have a session, we can start a channel by specifying the desired profile URI in a call to \texttt{startChannel} on the session.

The \texttt{sendMSG} method on the channel allows us to send a message on the channel. We need to specify a \texttt{OutputDataStream} and a \texttt{ReplyListener}. The \texttt{Reply} class implements this interface and allows to retrieve the reply from the server.



\subsection{Analysis}
\emph{BEEP} provides a framework for creating new application protocols. It fulfills our communication requirement pretty well, but not the discovery requirement. Other technologies could be used to discover shared documents, for instance \emph{Bonjour}.

\subsubsection{Positive Points}
\emph{BEEP} makes it relatively easy to create new protocols. It solves a lot of common problems, for instance multiplexing multiple logical channels onto one connection.

The message format used in channels is completely up to the application. It could be serialized Java objects (whereby one would loose language independence), XML messages or some binary format. There are implementations for Java, C, TCL, Ruby and others. So \emph{BEEP} would allow us to achieve interoperability with other collaborative editors programmed in other programming languages (unless, of course, we would use serialized Java objects as messages).

\subsubsection{Negative Points}
Creating \emph{BEEP} profiles is a low-level task. One has to deal with streams, bits and bytes. 

We would need to develop a protocol specification. All the messages and possible responses would need to be defined (e.g. as XML messages).

\subsubsection{Open Points}
There are no open points, besides defining the protocol and its messages.



\subsection{APEX} 
\emph{APEX} is a \emph{BEEP} profile. On top of \emph{BEEP} it adds service discovery, application-level addressing, presence information, and permission management. In other words, it defines how to find a person with whom you are interested in communicating, regardless of where they are, not unlike an electronic mail address does. It also allows discovery of presence information, meaning that one can be notified when a coworker becomes available for a video conference. Finally, it provides a standard mechanism for permissions, specifying a service and database for defining permissions associated with user names, applications and transactions.

From the general description this sound very much like what we need. It covers both discovery and communication. Unfortunately there does not seem to be a \emph{Java} implementation for the \emph{APEX} profile. The corresponding protocol specification is not finalized.



\subsection{Resources}
\begin{itemize}
 \item \href{http://www.beepcore.org/}{official beepcore website}
 \item \href{http://www-106.ibm.com/developerworks/xml/library/x-beep/index.html}{A bird's-eye view on beep}
 \item \href{http://www-128.ibm.com/developerworks/xml/library/x-beep2.html}{A worm's eye view on beep}
\end{itemize}

\section{BEEP}
\emph{BEEP} stands for Blocks Extensible Exchange Protocol. It is a protocol framework for building application protocols. Many Internet protocols reinvent a set of basic functions. The most common include:

\begin{itemize}
 \item Framing: separating each request from the next
 \item Matching responses to requests
 \item Pipelining: permitting multiple outstanding requests
 \item Multiplexing: permitting multiple asynchronous requests
 \item Reporting errors
 \item Negotiating encryption
 \item Negotiating authentication
\end{itemize}

\emph{BEEP} specifies reusable tools for all these functions, instead of requiring the same decisions to be made over again for each new application. It provides a framework that integrates existing Internet standards for encryption and authentication and new standards for connection management. A networked application needs merely to supply the \emph{important} part, the part that distinguishes it from other applications, as a \emph{BEEP} profile. The mundane parts are the same for all protocols, and therefore can be coded as a library, freeing the application designer to focus on the interesting bits.

\subsection{Why BEEP is Good}
\begin{itemize}
 \item It solves once a number of problems common to most connection-oriented application protocols.
 \item It allows a single application to operate in multiple roles, and even multiple roles simultanously.
 \item It is efficient and requires only a small overhead.
 \item It is low-risk because it is grounded in established Internet practice and it is standardized by the IETF.
\end{itemize}

\subsection{When BEEP is Bad}
Of course, there are a number of situations in which the use of \emph{BEEP} would be inappropriate. 
\begin{itemize}
 \item If a protocol already exists and is widely deployed, there is little sense in rewriting it to take advantage of \emph{BEEP}.
 \item If a protocol is not unicast, \emph{BEEP} is an inappropriate choice of transport (for instance multicast video streaming).
 \item \emph{BEEP} is inappropriate for real-time protocols.
\end{itemize}

\subsection{BEEP Concepts}
\emph{BEEP} is a peer-to-peer protocol in the sense that there is no notion of client or server. For convenience we'll refer to the peer that starts a connection as the \emph{initiator}, and the peer accepting the connection as the \emph{listener}. When a connection is established between the two, a \emph{BEEP} session is created.

\subsubsection{Channels}
All communication in a session happens within one or more \emph{channels}. The peers require only one IP connection, which is then multiplexed to create channels. The nature of communication possible within that channel is determined by the \emph{profiles} it supports (each channel may have one or more).

The first channel, channel 0, has a special purpose. It supports the \emph{BEEP} management profile, which is used to negotiate the setup of further channels. The supported profiles determine the precise interaction between the peers in a particular channel. Defining a protocol in \emph{BEEP} comes down to the definition of profiles.

\subsubsection{Profiles}
After the establishment of a session, the initiator asks to start a channel for the particular profile or set of profiles it whishes to use. If the listener supports the profile(s), the channel will be created. Profiles themselves take one of two forms: those for initial tuning and those for data exchange.

Tuning profiles, set up at the start of communication, affect the rest of the session in some way. For instance, requesting the TLS profile ensures that channels are encrypted using Transport Layer Security. Other tuning profiles perform steps such as authentication.

Data exchange profiles set expectations between the two peers as to what sort of exchanges will be allowed in a channel, similar to the was Java interfaces set expectations between interacting objects as to what methods are available. A profile is identified by a URI.

\subsubsection{Types of Data}
\emph{BEEP} puts no limits on the kind of data a channel can carry. It uses the MIME standard to support payloads of arbitrary type.


\subsection{Prototype}
\subsubsection{Echo Server}
The \emph{BEEP} implementation of an echo server can be found in \texttt{ch.iserver.ace.net.beep.EchoServer}. The code is straighforward. First a profile registry (\texttt{org.beepcore.beep.core.ProfileRegistry}) has to be created. All the supported profiles (\texttt{org.beepcore.beep.profile.Profile}) have to be added to the created profile registry by calling \texttt{addStartChannelListener}). The profiles are registered with an URI that is used as a key to specify a specific profile when creating a new channel.

Once the registry is set up and the profiles added, the server can start listening for client connections. The class \texttt{org.beepcore.beep.transport.tcp.TCPSessionCreator}'s \texttt{listen} method is used for this purpose. The server does not need to do anything special with the returned session as all the communication is handled by the selected profile.

The echo server uses the existing \texttt{org.beepcore.beep.profile.echo.EchoProfile} profile. This profile implements the echo behavior.

\subsubsection{Echo Client}
The echo client (see \texttt{ch.iserver.ace.net.beep.EchoClient}) connects to the echo server and sends echo requests using the BEEP echo profile. First a session has to be created to the server. The method \texttt{initiate} in \texttt{org.beepcore.beep.transport.tcp.TCPSessionCreator} is used for that purpose. Once we have a session, we can start a channel by specifying the desired profile URI in a call to \texttt{startChannel} on the session.

The \texttt{sendMSG} method on the channel allows us to send a message on the channel. We need to specify a \texttt{org.beepcore.beep.core.OutputDataStream} and a \texttt{org.beepcore.beep.core.ReplyListener}. The \texttt{org.beepcore.beep.lib.Reply} implements this interface and allows to retrieve the reply from the server.


\subsection{APEX}
\emph{APEX} is a \emph{BEEP} profile. On top of \emph{BEEP} it adds service discovery, application-level addressing, presence information, and permission management. In other words, it defines how to find a person with whom you are interested in communicating, regardless of where they are, not unlike an electronic mail address does. It also allows discovery of presence information, meaning that one can be notified when a coworker becomes available for a video conference. Finally, it provides a standard mechanism for permissions, specifying a service and database for defining permissions associated with user names, applications and transactions.

There does not seem to be a \emph{Java} implementation for the \emph{APEX} profile, although it does sound interesting as it provides both discovery (presence information) and communication.


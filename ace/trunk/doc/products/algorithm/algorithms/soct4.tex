\subsection{SOCT4}
\label{algo:soct4}

In \emph{SOCT4} (\cite{suleiman00}) as in \emph{SOCT3}, the operations are ordered globally using a timestamp given by a sequencer. They are then delivered on each site in this order thanks to the sequential reception. The originality of \emph{SOCT4} comes from the fact that inclusion transformation (called forward transposition) that takes into account concurrent operations are now made by the generator sites of the operations. According to \cite{suleiman00} this results in three major advantages:

\begin{enumerate}
 \item the receiver site does not have to separate history anymore; 
       thus backward transposition becomes unecessary
 \item the received operation can be stored as it is in the history
       without further transformation
 \item state vectors are no longer needed
\end{enumerate}

To achieve this, the broadcast of an operation must be deferred until it has been assigned a timestamp and all the operations which precede it according to the timestamp order have been received and executed. As usual, local operations are executed immediately without delay.


\subsubsection{Properties}
\begin{itemize}
 \item does not need any state vectors
 \item architecture: fully replicated
 \item uses a unique global ordering to abandon TP2 (by using a sequencer)
 \item no exclusion transformation needed
 \item no known user undo algorithm
\end{itemize}


\subsubsection{Sequencer}
\label{sequencer}
As noted before, both \emph{SOCT3} and \emph{SOCT4} use sequencer to globally serialize operations. In \emph{SOCT4} operations are broadcast sequentially. This makes collaboration difficult when the propagation delay of an operation on the network is high. This characteristic makes \emph{SOCT4} particularly adapted to fast networks.

More information about sequencers can be found in \cite{reed79}. Various methods for implementing sequencers are described in \cite{lelann78} (circulating sequencers) and \cite{banino79} (replicated sequencers).

\subsection{TIBOT}
\label{algo:tibot}

The \emph{TIBOT} algorithm presents a novel interaction model that is based on time intervals. It was presented in 2004 by Rui Li, Du Li and Chengzheng Sun in \cite{tibot}. In this summary, firstly some necessary concepts are introduced. Secondly the consistency control algorithm is explained.

\subsubsection{Concepts}
\paragraph{Time intervals based on logical clocks}
Every site in the replicated architecture maintains a linear logical clock. All clocks are initialized to a common value. A time interval is the period between two consecutive clock ticks. Time interval lengths are assumed to be the same. Every operation is timestamped by the current local clock value to indicate which time interval it belongs to. Function $TI(O)$ denotes the time interval of operation $O$. Two operations, $O_{i}$ and $O_{j}$, are in the same time interval iff $TI(O_{i}) = TI(O_{j})$, no matter whether $i = j$ or not.

\paragraph{Total ordering and operation context}
The total ordering is defined by means of the time interval function.  In addition, \emph{TIBOT} uses a linear history buffer in which the operations are stored in the total order at each site. The notion of operation context is the same as defined in \ref{definitions}. The operations are allowed to be executed in any order as long as their causality is preserved and the following synchronization rules are observed.

\paragraph{Propagation and synchronization rules}
\begin{itemize}
 \item Propagation rule 1: A local operation is propagated only when the current time interval is over and the operation has been transformed with all operations that have been executed in earlier time intervals. That is, a local operation always needs to be transfomred against all operations generated in previous time intervals before it can be propagated. Hence operation propagation is normally delayed.
 \item Propagation rule 2: When a time interval is over, if no operation has been generated locally, a control message is propagated to notify other sites that there was no operation generated in the past time interval. 
 \item Synchronization rule 1: Any remote operation received during a time interval can be synchronized only when (1) the local time interval is over; (2) the time interval of this operation is the same as the current local time interval; and (2) all operations totally preceding this operation have been executed.
 \item Synchronization rule 2: When a remote operation $O$ is to be executed, all operations that have been executed but out of the total order must be first undone, then perform $O$, and then redo these undone operations in sequence. These operations to be done or redone need to be transformed before they are executed (undo/do/redo scheme). Local operations are always executed immediately once generated.
 \item Synchronization rule 3: Operations with the same timestamp and site id are always synchronized as an entirety (i.e. are processed as entirety in transformations against local operations (concept of group operation)).
\end{itemize}
The above rules reduce the complexity of communication and interaction compared with other consistency models for group editors. E.g. ET is not necessary in this model.

\subsubsection{The TIBOT control algorithm}
Due to the rules defined above and the properties derived from these rules (see \cite{tibot}), the concurrency problem is simplified in this model. It can be reduced (from the three cases discussed in \cite{sun98b}) to the following two cases. Assume that O is a execution-ready operation. $[O_{1},O_{2},...,O_{n}]$ are a sequence of operations that have been executed locally and stored in the local history buffer (HB). Suppose HB = $[O_{1},O_{2},O_{3}]$.
\begin{itemize}
 \item Case 1: $O_{1} \rightarrow O, O_{2} \rightarrow O, O_{3} \rightarrow O$ \\
 All operations in HB are preceding O. Therefore O' = O, i.e., no transformation is needed.
 \item Case 2: $O_{1} \rightarrow O, O_{2} \parallel O, O_{3} \parallel O$ \\
 Operations preceding O are stored in HB before operations that are independent of O. Then O' is obtained by transforming O against concurrent operations, $O_{2}$ and $O_{3}$, in sequence.
\end{itemize}

The top-level control algorithm \emph{TIBOT} is based on these two cases and the undo/do/redo scheme. \emph{TIBOT} is called only when a remote group operation is ready for synchronization.


\subsubsection{Properties}
\begin{itemize}
 \item simplified and efficient approach based on time intervals instead of state vectors
 \item uses only IT
 \item is free from TP2
 \item implementation of time interval concept not clear
 \item architecture: replicated, multicast
\end{itemize}
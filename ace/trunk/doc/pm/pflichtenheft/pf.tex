\documentclass[11pt,a4paper]{article}
\usepackage[T1]{fontenc}
\usepackage{lmodern}
\usepackage[latin1]{inputenc}
\usepackage{ngerman}
\usepackage{a4wide}
\usepackage[dvips]{graphicx}

\usepackage{ace}

\usepackage[
pdfauthor={ACE Development Team},
pdftitle={Pflichtenheft},
pdfcreator={pdftex},
]{hyperref}

\begin{document}

\title{Pflichtenheft}
\author{Mark Bigler, Lukas Zbinden und Simon Raess}

\maketitle
\newpage
\tableofcontents
\newpage

\section{Einleitung}
Das Projekt ACE entwickelt einen kollaborativen, plattformunabh�ngigen Editor. Diese Applikation soll mehreren Personen erm�glichen, ein Textdokument gemeinsam zu bearbeiten. Dabei arbeitet jede Person mit dem Editor an einem eigenen Computer. Alle Teilnehmer sind �ber ein Netzwerk verbunden und sehen jederzeit den gleichen Dokumentinhalt. Wenn jemand der Gruppe eine �nderung im Dokument vornimmt, wird dies in Echtzeit und synchron allen anderen Benutzern angezeigt. Jeder Benutzer hat dadurch den �berblick �ber alle �nderungen im Dokument. Dieser Editor erm�glicht zum Beispiel ein gemeinsames Brainstorming von mehreren Personen, welche sich an verschiedenen Orten befinden. 

\subsection{Zweck des Dokumentes}
Das vorliegende Dokument beschreibt die Ziele, welche mit der angestrebten L�sung zu erreichen sind sowie die Anforderungen und W�nsche an das zuk�nftige System.

\section{Ausgangslage}
Das gemeinsame Entwerfen eines elektronischen Dokumentes (z.B. Textdatei), wobei jeder Beteiligte mit seinem eigenen Computer arbeitet, ist noch praktisch unbekannt. Diese Technologie birgt ein grosses Anwendungspotenzial. Der zu entwickelnde Editor ACE soll komplett neuartige und effiziente Editier-M�glichkeiten bieten und dadurch neue Wege der Zusammenarbeit er�ffnen. Bis heute existiert keine marktreife, plattformunabh�ngige Applikation dieser Art.

\section{Ist-Zustand}
Die dem kollaborativen Editor zugrunde liegende Theorie entstammt aus dem Forschungsgebiet der "Computer Supported Cooperative Work - CSCW". Applikationen aus diesem Bereich werden auch mit Groupware bezeichnet. Seit Mitte der 90er Jahren sind zahlreiche Forschungsarbeiten geschrieben worden. Die meisten davon beinhalten theoretische �berlegungen, so zum Beispiel mathematische Beschreibungen oder Beweise zu Synchronisationsalgorithmen. Viel Theoriewissen wurde erarbeitet. Dieses Know-How soll nun in die konkrete Implementation eines kollaborativen Editors einfliessen um daraus eine hochentwickelte, konkurrenzf�hige Applikation auf den Markt zu bringen. 

\section{Ziele}
Die nachfolgenden Zielsetzungen gelten f�r die Semesterarbeit.
 \begin{center}
  \begin{tabular}{|l|c|}
   \hline
   Beschreibung     &     Priorit�t    \\
   \hline
    Implementation eines funktionierenden Synchronisationsalgorithmus, dies beinhaltet:                    &        1        \\
    Definiertes, konsistentes Verhalten des Algorithmus bei allen m�glichen Use Cases                       &        1        \\
    Identischer Dokumentinhalt bei allen Editorinstanzen einer Gruppe                                       &        1         \\
    Cursor-Positionen der verschiedenen Teilnehmer voneinander unabh�ngig                                   &        1         \\
    Dynamisches Erkennen aller freigegebenen Dokumenten innerhalb eines Netzwerkes                          &        2         \\
    Freigeben von Dokumenten in Read-Only oder Read-Write Modus                                             &        2         \\
    Sich als Reader oder Editor einem Dokument anschliessen                                                 &        2         \\
    Einfache GUI Funktionen
   \hline
  \end{tabular}
 \end{center}

\subsection{Implementation eines funktionierenden Synchronisationsalgorithmus}

\subsection{Definiertes, konsistentes Verhalten des Algorithmus bei allen m�glichen Use Cases}

\subsection{Identischer Dokumentinhalt bei allen Editorinstanzen einer Gruppe}

\subsection{Cursor-Positionen der verschiedenen Teilnehmer voneinander unabh�ngig}

\subsection{Dynamisches Erkennen aller freigegebenen Dokumenten innerhalb eines Netzwerkes}

\subsection{Freigeben von Dokumenten in Read-Only oder Read-Write Modus}

\section{Anforderungen}
\subsection{Architektur}
Die Architektur wird mit dem Peer-to-Peer Modell erarbeitet. Die Applikation soll ohne einen zentralen Server funktionieren.

\subsection{Schnittstellen}
Durch die Offenlegung des internen Kommunikations-Protokolls von ACE haben andere Applikationsanbieter die M�glichkeit, direkt mit dem Editor zu kommunizieren. Dadurch ergeben sich beliebige Anwendungsm�glichkeiten.

\end{document}

\documentclass[11pt,a4paper]{article}
\usepackage[T1]{fontenc}
\usepackage{lmodern}
\usepackage[latin1]{inputenc}
\usepackage{ngerman}
\usepackage{a4wide}
\usepackage[dvips]{graphicx}

\usepackage{ace}

\usepackage[
pdfauthor={ACE Development Team},
pdftitle={Pflichtenheft},
pdfcreator={pdftex},
]{hyperref}

\begin{document}

\title{Pflichtenheft}
\author{Mark Bigler, Lukas Zbinden und Simon Raess}

\maketitle
\newpage
\tableofcontents
\newpage

\section{Einleitung}
Das Projekt ACE entwickelt einen kollaborativen, plattformunabh�ngigen Editor. Diese Applikation soll mehreren Personen erm�glichen, ein Textdokument gemeinsam zu bearbeiten. Dabei arbeitet jede Person mit dem Editor an einem eigenen Computer. Alle Teilnehmer sind �ber ein Netzwerk verbunden und sehen jederzeit den gleichen Dokumentinhalt. Wenn jemand der Gruppe eine �nderung im Dokument vornimmt, wird dies in Echtzeit und synchron allen anderen Benutzern angezeigt. Jeder Benutzer hat dadurch den �berblick �ber alle �nderungen im Dokument. Dieser Editor erm�glicht zum Beispiel ein gemeinsames Brainstorming von mehreren Personen, welche sich an verschiedenen Orten befinden. 

\subsection{Zweck des Dokumentes}
Das vorliegende Dokument beschreibt die Ziele, welche mit der angestrebten L�sung zu erreichen sind sowie die Anforderungen und W�nsche an das zuk�nftige System.

\section{Ausgangslage}
Das gemeinsame Entwerfen eines elektronischen Dokumentes (z.B. Textdatei), wobei jeder Beteiligte mit seinem eigenen Computer arbeitet, ist noch praktisch unbekannt. Diese Technologie birgt ein grosses Anwendungspotenzial. Der zu entwickelnde Editor ACE soll komplett neuartige und effiziente Editier-M�glichkeiten bieten und dadurch neue Wege der Zusammenarbeit er�ffnen. Bis heute existiert keine marktreife, plattformunabh�ngige Applikation dieser Art.

\section{Ist-Zustand}
Die dem kollaborativen Editor zugrunde liegende Theorie entstammt aus dem Forschungsgebiet der "'Computer Supported Cooperative Work - CSCW"'. Applikationen aus diesem Bereich werden auch mit Groupware bezeichnet. Seit Mitte der 90er Jahren sind zahlreiche Forschungsarbeiten geschrieben worden. Die meisten davon beinhalten theoretische �berlegungen, so zum Beispiel mathematische Beschreibungen oder Beweise zu Synchronisationsalgorithmen. Viel Theoriewissen wurde erarbeitet. Dieses Know-How soll nun in die konkrete Implementation eines kollaborativen Editors einfliessen um daraus eine hochentwickelte, konkurrenzf�hige Applikation auf den Markt zu bringen. 

\section{Ziele}
In der Semesterarbeit soll die Basis gelegt werden f�r die Implementation eines kollaborativen Texteditors im Rahmen der Diplomarbeit.
 \begin{center}
  \begin{tabular}{|l|c|}
   \hline
   Beschreibung     &     Priorit�t    \\
   \hline
    Aufbau von Knowhow im Bereich CSCW                          &        1         \\
    Bestehende Algorithmen evaluieren                           &        1         \\
    Algorithmus implementieren                                  &        1         \\
    Testframework f�r Algorithmus (Testf�lle)                   &        1         \\
    Konzept GUI                                                 &        2         \\
    Konzept Netzwerk/Kommunikation                              &        2         \\
    User Stories f�r kollaborativen Texteditor                  &        3         \\
   \hline
  \end{tabular}
 \end{center}

\subsection{Aufbau von Knowhow im Bereich CSCW}
Es geht darum, ein fundiertes Basiswissen im Bereich des CSCW aufzubauen. Das Ziel ist, eine �bersicht �ber den aktuellen Forschungsstand und �ber die wichtigsten Errungenschaften in diesem Gebiet zu gewinnen. Das angeeignete Know-How soll beim Evaluieren der bestehenden Synchronisationsalgorithmen zum tieferen Verst�ndnis beitragen und ein sachgerechtes Beurteilen erm�glichen.

\subsection{Bestehende Algorithmen evaluieren}
Die Forschung hat seit anfangs der 90er Jahre zahlreiche Synchronisations-Algorithmen in formaler Form entwickelt. Das Ziel ist eine �bersicht zu erabeiten �ber die besten bis heute bekannten Algorithmen. Jeder evaluierte Algorithmus soll prinzipiell verstanden werden. Die �bersicht soll Vor- und Nachteile aufzeigen und es erm�glichen, den am besten geeigneten Algorithmus f�r einen kollaborativen Texteditor zu bestimmen.

\subsection{Algorithmus implementieren}
Nach Evaluation eines bestimmten Algorithmus soll dieser implementiert werden (als Synchronisationsengine). Die Implementation beinhaltet ebenfalls, m�gliche Nachteile respektive Schwachstellen des gew�hlten Algorithmus zu beheben durch eigens entwickelten L�sungen oder Kombination mit anderen Algorithmen. Die Implementation des Algorithmus wird mit dem parallel erstellten Testframework getestet.

\subsection{Testframework f�r Algorithmus}
Parallel zur Entwicklung des Synchronisationsalgorithmus soll ein Framework zum Austesten erstellt werden. Diese soll das sorgf�lltige Austesten des Algorithmus mit verschiedenen, klar definierbaren (z.B. mit Hilfe von XML-Dateien) Abl�ufen erm�glichen.


\subsection{Konzept GUI}
In erster Linie geht es darum, zu sehen was mit den verschiedenen Textkomponenten von Java alles machbar ist. Es muss m�glich sein mehrere Cursor, ausw�hlbare und anders eingef�rbte Textstellen in einer solchen Textkomponente darzustellen. Weiter m�ssen spezielle Auflistungskomponenten analysiert werden, mit welchen alle aktiven Benutzer, deren freigegebene Dokumente und Zugriffsrechte angezeigt werden k�nnen. Diese Komponenten sollten alle einfach und intuitiv zu bedienen sein.

\subsection{Konzept Netzwerk/Kommunikation}
Die verschiedenen Kommunikationsmechanismen, speziell das gegenseitige Auffinden anderer Netzwerkteilnehmer nach dem Rendez-Vous-Prinzip, sollen unter die Lupe genommen werden.


\subsection{User Stories f�r kollaborativen Texteditor}
Es soll eine Sammlung von User Stories (Anwendungsm�glichkeiten aus Sicht des Benutzers) f�r einen kollaborativen Texteditor erstellt werden. Diese Sammlung bildet dann eine Ideen-Grundlage f�r die Implementation von Funktionen des in der Diplomarbeit zu entwickelnden Texteditors.


\section{Anforderungen}
\subsection{Architektur}
Die Architektur wird mit dem Peer-to-Peer Modell erarbeitet. Die Applikation soll ohne einen zentralen Server funktionieren.

\subsection{Schnittstellen}
Durch die Offenlegung des internen Kommunikations-Protokolls von ACE haben andere Applikationsanbieter die M�glichkeit, direkt mit dem Editor zu kommunizieren. Dadurch ergeben sich beliebige Anwendungsm�glichkeiten.

\end{document}

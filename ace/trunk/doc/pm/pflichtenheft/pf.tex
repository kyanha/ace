\documentclass[11pt,a4paper]{article}
\usepackage[T1]{fontenc}
\usepackage{lmodern}
\usepackage[latin1]{inputenc}
\usepackage{ngerman}
\usepackage{a4wide}
\usepackage[dvips]{graphicx}

\usepackage{ace}

\usepackage[
pdfauthor={ACE Projekt Team},
pdftitle={Pflichtenheft},
pdfcreator={pdftex},
]{hyperref}

\begin{document}

\title{Pflichtenheft}
\author{Mark Bigler, Lukas Zbinden und Simon Raess}

\maketitle
\newpage
\tableofcontents
\newpage

\section*{Versionskontrolle}

\begin{table}[!h]
 \begin{tabular}{|l|l|l|l|}
  \hline
  \headercol{0.6in}{Version}         & 
  \headercol{0.8in}{Datum}           &
  \headercol{1.2in}{Verantwortlich}  & 
  \headercol{2.8in}{Bemerkungen}     \\
  \hline
  0.1         & 15.03.2005  & rasss           &  Erste Version \\
  \hline
  0.2         & 22.03.2005  & rasss zbinl     &  �berarbeitung \\
  \hline
  0.3         & 30.03.2005  & zbinl           &  Projektspezifisches Vorgehensmodell \\
  \hline
  0.4         & 30.03.2005  & rasss zbinl     &  �berarbeitung Punkte 2, 3 und 4 \\
  \hline
  0.5         & 05.04.2005  & zbinl           &  �berarbeitung Appendix A.1 \\
  \hline
  0.6         & 06.04.2005  & zbinl           &  Anpassungen Punkte 2, 3 und 4 \\
  \hline
  0.7         & 06.04.2005  & Projektteam     &  Review \\
  \hline
  0.8         & 13.04.2005  & Projektteam     &  Letzte Anpassungen \\
  \hline
  1.0         & 13.04.2005  & Projektteam     &  Freigabe \\
  \hline
 \end{tabular}
 \caption{Versionskontrolle}
 \label{Versionskontrolle}
\end{table}

\begin{table}[!h]
 \begin{tabular}{|l|l|l|l|l|}
  \hline
  \headercol{0.9in}{}            & 
  \headercol{0.9in}{Stelle}      & 
  \headercol{0.8in}{Datum}       & 
  \headercol{0.6in}{Visum}       & 
  \headercol{2.0in}{Bemerkungen} \\
  \hline
  \textbf{Freigegeben}   & Projektteam &       &       &             \\
  \hline
  \textbf{Genehmigt}     &             &       &       &             \\
  \hline
 \end{tabular}
 \caption{Pr�fung/Genehmigung}
 \label{Pr�fung/Genehmigung}
\end{table}

\newpage

\section{Einleitung}
Im Projekt \ace soll ein kollaborativer, plattformunabh�ngiger Editor entwickelt werden. Diese Applikation erm�glicht mehreren Personen, ein Textdokument gemeinsam zu bearbeiten. Dabei arbeitet jede Person mit dem Editor an einem eigenen Computer. Alle Teilnehmer sind �ber ein Netzwerk verbunden und sehen jederzeit den gleichen Dokumentinhalt. Wenn jemand der Gruppe eine �nderung im Dokument vornimmt, wird dies in Echtzeit und synchron allen anderen Benutzern angezeigt. Jeder Benutzer hat dadurch den �berblick �ber alle �nderungen im Dokument. Dieser Editor erm�glicht zum Beispiel ein gemeinsames Brainstorming von mehreren Personen, welche sich an verschiedenen Orten befinden. 

\subsection{Zweck des Dokumentes}
Das vorliegende Dokument beschreibt die Ziele, welche mit der angestrebten L�sung zu erreichen sind sowie die Anforderungen und W�nsche an das zuk�nftige System.

\section{Ausgangslage}
Das gemeinsame Entwerfen eines elektronischen Dokumentes (z.B. Textdatei), wobei jeder Beteiligte mit seinem eigenen Computer arbeitet, ist noch praktisch unbekannt. Diese Technologie birgt ein grosses Anwendungspotenzial. Der zu entwickelnde Editor \ace soll komplett neuartige und effiziente Editier-M�glichkeiten bieten und dadurch neue Wege der Zusammenarbeit er�ffnen. Bis heute existiert keine marktreife, plattformunabh�ngige Applikation dieser Art.

\section{Ist-Zustand}
Die dem kollaborativen Editor zugrunde liegende Theorie entstammt aus dem Forschungsgebiet der "'Computer Supported Cooperative Work - CSCW"'. Applikationen aus diesem Bereich werden auch als "'Groupware"' bezeichnet. Seit Mitte der 90er Jahren sind zahlreiche Forschungsarbeiten geschrieben worden. Die meisten davon beinhalten theoretische �berlegungen, so zum Beispiel mathematische Beschreibungen oder Beweise zu Synchronisationsalgorithmen. Viel Theoriewissen wurde erarbeitet. Dieses Know-How soll nun in die konkrete Implementation eines kollaborativen Editors einfliessen um daraus eine hochentwickelte, konkurrenzf�hige Applikation auf den Markt zu bringen. 

\section{Ziele}
In der Semesterarbeit soll die Basis gelegt werden f�r die Implementation eines kollaborativen und plattformunabh�ngigen Texteditors im Rahmen der Diplomarbeit.

 \begin{table}[!ht]
  \begin{tabular}{|l|c|}
   \hline
   \headcol{4.5in}{Beschreibung}                                & \headcol{1in}{Priorit�t}    \\
   \hline
    Aufbau von Knowhow                                          &        1                    \\
    Evaluation bestehender Algorithmen                          &        1                    \\
    Implementation Algorithmus                                  &        1                    \\
    Testframework f�r Algorithmus                               &        1                    \\
    Konzept GUI                                                 &        2                    \\
    Konzept Netzwerk/Kommunikation                              &        2                    \\
    User Stories f�r kollaborativen Texteditor                  &        3                    \\
   \hline
  \end{tabular}
  \caption{Projektziele}
  \label{Projektziele}
 \end{table}

\subsection{Aufbau von Knowhow}
Es geht darum, ein fundiertes Basiswissen im Bereich des CSCW aufzubauen. Das Ziel ist, eine �bersicht �ber den aktuellen Forschungsstand und �ber die wichtigsten Errungenschaften in diesem Gebiet zu gewinnen. Das angeeignete Know-How soll beim Evaluieren der bestehenden Synchronisationsalgorithmen zum tieferen Verst�ndnis beitragen und ein sachgerechtes Beurteilen erm�glichen.

\subsection{Evaluation bestehender Algorithmen}
Die Forschung hat seit anfangs der 90er Jahre zahlreiche Synchronisations-Algorithmen entwickelt. Das Ziel ist eine �bersicht zu erabeiten �ber die besten bestehenden Algorithmen. Jeder evaluierte Algorithmus soll prinzipiell verstanden werden. Die �bersicht soll Vor- und Nachteile aufzeigen und es erm�glichen, den am besten geeigneten Algorithmus f�r einen kollaborativen Texteditor zu bestimmen.

\subsection{Implementation Algorithmus}
Nach Evaluation der Algorithmen soll einer oder eventuell mehrere implementiert werden. Die Implementation des Algorithmus wird mit dem parallel erstellten Testframework gepr�ft.

\subsection{Testframework f�r Algorithmus}
Parallel zur Entwicklung des Synchronisationsalgorithmus soll ein Testframework erstellt werden. Dieses erm�glicht das sorgf�lltige Austesten des Algorithmus mit verschiedenen, klar definierbaren Abl�ufen.

\subsection{Konzept GUI}
In erster Linie soll eine Evaluation von Textkomponenten in Java erfolgen. Das Ziel ist herauszufinden, welche sich am besten f�r einen kollaborativen Texteditor eignen. Es muss m�glich sein, verschiedene Benutzer respektive deren Aktionen in einer Textkomponente darzustellen. 

\subsection{Konzept Netzwerk/Kommunikation}
Zwei Themen m�ssen erarbeitet werden: Das dynamische Auffinden aller vorhandenen Instanzen des Editors in einem Netzwerk sowie der Nachrichtenaustausch zwischen den Instanzen. Es sollen die in diesem Themengebiet vorhandenen Technologien studiert werden. Daraus entsteht ein f�r einen kollaborativen Texteditor angepasstes Konzept. 

\subsection{User Stories f�r kollaborativen Texteditor}
Das Ziel ist, eine Sammlung von User Stories (Anwendungsm�glichkeiten aus Sicht des Benutzers) f�r einen kollaborativen Texteditor zu erstellen. Diese Sammlung bildet dann eine Ideen-Grundlage f�r die Implementation von Funktionen des in der Diplomarbeit zu entwickelnden Texteditors.

\section{Anforderungen}
\subsection{Architektur}
Die Architektur wird basierend auf dem Peer-to-Peer Modell entwickelt. Die Applikation soll dadurch ohne einen zentralen Server funktionieren und eine einfache Installation erm�glichen.

\subsection{Schnittstellen}
Durch die Offenlegung des internen Kommunikations-Protokolls von \ace haben andere Applikationsentwickler die M�glichkeit, direkt mit dem Editor zu kommunizieren. Dadurch entstehen beliebige Erweiterungsm�glichkeiten mit dem kollaborativen Texteditor.

\end{document}
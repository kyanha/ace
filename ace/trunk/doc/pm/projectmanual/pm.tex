\documentclass[11pt,a4paper]{article}
\usepackage[T1]{fontenc}
\usepackage{lmodern}
\usepackage[latin1]{inputenc}
\usepackage{ngerman}
\usepackage{a4wide}
\usepackage[dvips]{graphicx}
\usepackage{listings}

\usepackage{ace}

\usepackage[
pdfauthor={ ACE Projekt Team },
pdftitle={ Projekthandbuch },
pdfcreator={pdftex},
]{hyperref}

\begin{document}

\tableofcontents
\newpage
\listoftables
\listoffigures
\newpage

\section*{Versionskontrolle}

\begin{table}[!h]
 \begin{tabular}{|l|l|l|l|}
  \hline
  \headercol{0.6in}{Version}         & 
  \headercol{0.8in}{Datum}           &
  \headercol{1.2in}{Verantwortlich}  & 
  \headercol{2.8in}{Bemerkungen}     \\
  \hline
  0.1         & 15.03.2005  & rasss           &  Erste Version \\
  \hline
  0.2         & 22.03.2005  & rasss zbinl     &  �berarbeitung \\
  \hline
  0.3         & 30.03.2005  & zbinl           &  Projektspezifisches Vorgehensmodell \\
  \hline
  0.4         & 30.03.2005  & rasss zbinl     &  �berarbeitung Punkte 2, 3 und 4 \\
  \hline
  0.5         & 05.04.2005  & zbinl           &  �berarbeitung Appendix A.1 \\
  \hline
  0.6         & 06.04.2005  & zbinl           &  Anpassungen Punkte 2, 3 und 4 \\
  \hline
  0.7         & 06.04.2005  & Projektteam     &  Review \\
  \hline
  0.8         & 13.04.2005  & Projektteam     &  Letzte Anpassungen \\
  \hline
  1.0         & 13.04.2005  & Projektteam     &  Freigabe \\
  \hline
 \end{tabular}
 \caption{Versionskontrolle}
 \label{Versionskontrolle}
\end{table}

\begin{table}[!h]
 \begin{tabular}{|l|l|l|l|l|}
  \hline
  \headercol{0.9in}{}            & 
  \headercol{0.9in}{Stelle}      & 
  \headercol{0.8in}{Datum}       & 
  \headercol{0.6in}{Visum}       & 
  \headercol{2.0in}{Bemerkungen} \\
  \hline
  \textbf{Freigegeben}   & Projektteam &       &       &             \\
  \hline
  \textbf{Genehmigt}     &             &       &       &             \\
  \hline
 \end{tabular}
 \caption{Pr�fung/Genehmigung}
 \label{Pr�fung/Genehmigung}
\end{table}



\section{Einleitung}

\subsection{Zweck des Dokuments}

Das Projekthandbuch dient als einheitliche Handlungsgrundlage f�r alle Projektbeteiligten und legt damit den allgemeing�ltigen technischen und organisatorischen Rahmen fest. 

Dieses Dokument ist soweit wie m�glich als statisches Dokument zu f�hren. Es ist jedoch zu Beginn und am Schluss jeder Phase zu �berpr�fen und an die neuen Erkentnisse anzupassen.


\section{Projektbeschreibung}

Das gemeinsame Editieren von Dokumenten kann eine grosse Herausforderung sein. Versionierungs Systeme wie Subversion und CVS k�nnen einer Gruppe helfen, ein Dokument gemeinsam zu bearbeiten. Diese Systeme erm�glichen aber nicht ein gemeinsames Bearbeiten in Echtzeit. Genau an dieser Stelle setzt das Projekt \emph{ACE} an. Es soll ein Texteditor implementiert werden, der das gleichzeitige Bearbeiten des gleichen Dokumentes in Echtzeit erm�glicht.

\subsection{Ausgangssituation}
Heutzutage besitzt praktisch jeder Mensch in den Industriel�ndern einen Computer oder hat einen Zugang dazu. Verfassen von Texten, e-Mails und das Verwenden des World Wide Web geh�ren f�r die meisten Menschen zum Alltag. Doch gerade im Bereich Zusammenarbeit erleichtern uns Computer die Arbeit oft nicht. Im Gegenteil, oft verhindern sie ein intuitives Zusammenarbeiten. Allgemein ist man der �berzeugung, dass Anwendungen, welche die Zusammenarbeit f�rdern und erleichtern, ein grosses Potential haben. Insbesondere in einer Zeit wo SMS, Instant Messaging, Mobil- und Internet-Telefonie zum Alltag geh�ren.

\subsection{Ziele}
In der Semesterarbeit soll die Basis gelegt werden f�r die Implementation eines kollaborativen Texteditors im Rahmen der Diplomarbeit.

\begin{table}[!h]
 \begin{center}
 \begin{tabular}{|l|l|l|}
 \hline
 \headercol{0.5in}{Ziele}  & 
 \headercol{1in}{Priorit�ten} &
 \headercol{4.1in}{Beschreibung} \\
 
 \hline
 1     & 1 & Aufbau von Knowhow im Bereich CSCW \\
 \hline
 2     & 1 & Evaluation bestehender Algorithmen \\
 \hline
 3     & 1 & Implementation Algorithmus \\
 \hline
 4     & 1 & Testframework f�r Algorithmus \\
 \hline
 5     & 2 & Konzept GUI \\
 \hline
 6     & 2 & Konzept Netzwerk/Kommunikation \\
 \hline
 7     & 3 & User Stories f�r kollaborativen Texteditor \\
 \hline
 \end{tabular}
 \end{center}
 \caption{Ziele}
 \label{Ziele}
\end{table}


\subsection{Vorgehensstrategie}

In dem Semesterprojekt geht es vorallem darum, Wissen aufzubauen und dazu einen funktionierenden Synchronisationsalgorithmus zu entwickeln. Die Ziele sind teilweise unabh�ngig voneinander. Grunds�tzlich werden pro unabh�ngigem Teil die Phasen Voranalyse und Konzept durchlaufen. Beim Teilziel Synchronisationsalgorithmus werden zus�tzlich noch eine Implementierung und damit verbundene Tests erstellt. 


\section{Projektspezifisches Vorgehensmodell}

\subsection{Phase Initialisierung}
In der Phase Initialisierung werden folgende Ergebnisse erstellt:
\begin{itemize}
 \item Projektantrag
 \item Projekthandbuch
 \item Pflichtenheft
 \item Projektplan
\end{itemize}

\subsection{Phase Voranalyse}
In der Phase Voranalyse werden jeweils die folgenden Ergebnisse erstellt:
\begin{itemize}
 \item Bericht (m�gliche L�sungen)
\end{itemize}

\subsection{Phase Konzept}
In der Phase Konzept werden jeweils die folgenden Ergebnisse erstellt:
\begin{itemize}
 \item detaillierter Bericht Konzept
 \item Prototyp (kritische Teilsysteme untersuchen)
\end{itemize}


\section{Entscheidungspunkte und auszuliefernde Ergebnisse}

\subsection{Initialisierung}
\subsubsection{Ergebnisse}
\begin{itemize}
 \item Projektantrag 
 \item Projekthandbuch
 \item Pflichtenheft
 \item Projektplan
\end{itemize}
\subsubsection{Entscheidungspunkte}
\begin{itemize}
 \item keine
\end{itemize}

\subsection{Synchronisationsalgorithmus}
\subsubsection{Ergebnisse}
\begin{itemize}
 \item Bericht Synchronisationsalgorithmen
 \item Testframework Synchronsiationsalgorithmen
 \item Erfassen und Erstellen von Testf�llen
 \item Implementation Synchronisationsalgorithmus
\end{itemize}
\subsubsection{Entscheidungspunkte}
\begin{itemize}
 \item Auswahl Synchronisationsalgorithmus basierend auf Bericht
 \item Abschluss Implementation
\end{itemize}

\subsection{Konzept GUI/Netzwerk}
\subsubsection{Ergebnisse}
\begin{itemize}
 \item ein Bericht (m�gliche L�sungen, gew�hlte L�sung, Erfahrungen)
 \item Prototyp
\end{itemize}
\subsubsection{Entscheidungspunkte}
\begin{itemize}
 \item Auswahl L�sung
\end{itemize}

\subsection{User Stories}
F�r die Erfassung des Anwendungsumfangs sollen User Stories erfasst werden. 
\subsubsection{Ergebnisse}
\begin{itemize}
 \item User Stories Katalog
\end{itemize}
\subsubsection{Entscheidungspunkte}
\begin{itemize}
 \item keine
\end{itemize}



\section{Methoden und Werkzeuge}

\subsection{Dokumente}
Zum Erstellen aller Dokumente soll grunds�ztlich \LaTeX{} verwendet werden. Selbstverst�ndlich werden f�r Grafiken und Diagramme andere Anwendungen verwendet. Da es sich bei \LaTeX{} Dateien um Textdateien handelt, k�nnen wir SubEthaEdit (ein kollaborativer Editor f�r Mac) verwenden, um zur gleichen Zeit am gleichen Dokument zu arbeiten.

\subsection{Source Repository}
Subversion wird als Source Repository verwendet. Die URL f�r den Subversion Zugriff ist http://ace.iserver.ch:81/repos/ace/ace.

\subsection{Projektwebsite}
Von der Projektwebsite aus erreicht man alle wichtigen Werkzeuge. Die Projektwebsite ist erreichbar unter http://ace.iserver.ch. Die Projektwebsite dient auch dazu, das Produkt \emph{ACE} zu vermarkten. 

\subsection{Zeiterfassung}
Zur Erfassung der Arbeitszeit wird eine PHP Applikation, entwickelt von Herrn Bigler, verwendet. Dies wird es erm�glichen, am Ende des Projektes einige Angaben zu den geleisteten Arbeitsstunden zu machen.

\subsection{Trac}
Trac ist ein sogenannter \emph{issue tracker} mit einem integrierten Wiki. Das Wiki wird f�r den projektinternen Informationsaustausch verwendet. Der \emph{issue tracker} wird f�r die gezielte Erfassung von Bug Reports verwendet.

\subsection{Kalender}
Ein online verf�gbarer Projektkalender gibt Auskunft �ber alle anstehenden Sitzungen und andere Termine (http://ace.iserver.ch/phpicalendar/week.php).

\subsection{Entwicklung}
\begin{itemize}
 \item Eclipse (http://www.eclipse.org/)
 \item SubEthaEdit (http://www.codingmonkeys.de/subethaedit/)
 \item ant (http://ant.apache.org/)
\end{itemize}


\section{Standards und Richtlinien}

\subsection{Dokumentation}
Alle \LaTeX{} Dokumente sollen die Package ace.tex verwenden. Diese Datei findet man im Subversion Repository unter \texttt{/ace/trunk/doc/latex/ace.tex}. Eine Vorlage f�r alle \LaTeX{} Dokumente findet man im selben Verzeichnis (template.tex).

\subsection{Quellcode}
Die Dokumentation des Quellcodes erfolgt mit JavaDoc. Die Kommentare werden einheitlich in Englisch verfasst. Quellcodedateien werden mit folgendem Header, welcher unter \texttt{/ace/trunk/doc/templates/source.header} zu finden ist, versehen.

\lstset{language=Java}
\small{
\begin{verbatim}
/*
 * $Id$
 *
 * ace - a collaborative editor
 * Copyright (C) 2005 Mark Bigler, Simon Raess, Lukas Zbinden
 *
 * This program is free software; you can redistribute it and/or
 * modify it under the terms of the GNU General Public License
 * as published by the Free Software Foundation; either version 2
 * of the License, or (at your option) any later version.
 *
 * This program is distributed in the hope that it will be useful,
 * but WITHOUT ANY WARRANTY; without even the implied warranty of
 * MERCHANTABILITY or FITNESS FOR A PARTICULAR PURPOSE.  See the
 * GNU General Public License for more details.
 *
 * You should have received a copy of the GNU General Public License
 * along with this program; if not, write to the Free Software
 * Foundation, Inc., 59 Temple Place - Suite 330, Boston, MA  02111-1307, USA.
 */
\end{verbatim}
}

\subsubsection{Konventionen}
Die Quellcodedateien sollen gem�ss den Sun Coding Conventions (http://java.sun.com/docs/codeconv/) erstellt werden. Die Einr�ckung erfolgt mit Tabulator Zeichen (nicht mit Spaces).


\appendix

\section{Undo in multiuser collaborative applications}
\label{sect:undo}

The availability of undo in a multiuser collaborative application is valuable because features available in single-user applications should also be available in corresponding multi-user applications. However, supporting undo in a multiuser collaborative application is much more difficult than supporting undo in single-user interactive applications because of the interleaving of operations performed by multiple users in a collaborative computing environment.


\subsection{An Example}

To illustrate the difficulties, let us consider a collaborative editing session with two users and a shared text document containing the string '"abc'". Suppose user 1 issues the operation $O_{1} = Ins[2,X]$ to insert the character '"X'" at position 2 after the character '"a'". The resulting document state is '"aXbc'". After this, user 2 issues operation $O_{2} = Ins[2,Y]$ to insert character '"Y'" after the character '"a'" to get the resulting document state '"aYXbc'". Suppose user 1 wants to undo his last operation, i.e. $O_{1}$. Her intended effect is to remove the '"X'" thus resulting in a document state '"aYbc'". First, blindly picking the \emph{globally} last operation, i.e. $O_{2}$ will undo the wrong operation. Second, simply executing the inverse operation $\overline{O_{1}} = Del[2,X]$ to delete the '"X'" at position 2 in the current document state '"aYXbc'" will delete '"Y'" instead of '"X'".

This means that any group undo solution must meet the challenge of undoing operations in a nonlinear way and must be able to achieve the correct undo effect in a document state that has been changed by other users'' operations. It can be clearly seen that undo in a groupware system is highly related to operational transformations.


\subsection{Types of Undo}

Generally a user expects an undo to reverse their own last operation (\emph{local} undo) rather than the globally last operation (\emph{global} undo). So an undo framework for groupware systems needs to allow selection of an operation to undo based on who performed it. Undoing the globally last action can be problematic as just before one user presses undo, another user may issue an operation. The effect of the operation may get executed at the first users site just before he effectively invokes the undo action. This would not result in the undoing of an operation which the first user intended to undo.

Further an undo mechanism may be classified by whether it allows to undo an arbitrary operation (called \emph{selective} undo) or it is restricted to the \emph{chronological} order.

It is still an open question how to select operations to be undone in a selective undo system. To undo the chronoligically last operation of the local user, the single familiar undo menu or shortcut is sufficient as it is always clear wich operation has to be undone.


\subsection{Existing Undo Solutions}

\subsubsection{DistEdit System}
The DistEdit system \cite{prakash94} allows operations to be undone in any order. However, it may be that an operation is not undoable because it conflicts with a later executed operations. The conflicts occur if a later operation happens at the same index as the operation to undo. They managed to solve some conflicting situations, but still a few remained in the finished system.

\subsubsection{adOPTed undo}
In the \emph{adOPTed} algorithm (see \ref{algo:adopted}) undo is supported by operational transformation. It allows to undo the chronological last operation of a user (it is implemented only for the local user but would it would be possible to extend it to an arbitrary user). To undo an operation, an inverse of this operation is generated by a special \emph{mirror} operator. This inverse operation must be placed at a valid location in the corresponding dimension of the interaction graph. No conflict can occur in undoing any operation. Another special operator called \emph{folding} is used for a correct working system. See \cite{ressel99} for a detailed description of this undo system.

\subsubsection{GOTO undo}
Sun \cite{sun02b} described an undo mechanism for the \emph{REDUCE} prototype. It is using the \emph{GOTO} algorithm (see \ref{algo:goto}) for \emph{do} and an algorithm called \emph{ANYUNDO} for \emph{undo}. The algorithm allows undoing arbitrary operations, so it is a selective undo system. The algorithm is described in great detail along many possible problems and how they were solved in this algorithm.



\section{Vector Time}
\label{sect:vectortime}

Most algorithms use \emph{vector time} to determine causal relations. Each site maintains a state vector $v$ that has $n$ components. $n$ is the number of participating sites. $S$ is the local site. The $i$-th component of $v$ denoted as $v[i]$ represents the number of operations executed from site $i$ at site $S$ (the local site). 

\begin{defn}
  For two time vectors u, v \\
  $u \leq v$ iff $\forall i : u[i] \leq v[i]$ \\
  $u < v$ iff $u \leq v$ and $u \not= v$ \\
  $u \parallel v$ iff $\neg(u < v)$ and $\neg(v < u)$
\end{defn}

Notice that $\leq$ and $<$ are partial orders. The \emph{concurrency relation} $\parallel$ is reflexive and symmetric.

The above definition gives us a very simple method to decide whether two events $e$ and $e'$ are causally related or not: We take their timestamps (vector times) and check whether $C(e) < C(e')$ or $C(e') < C(e)$ where $C(x)$ determines the timestamp of $x$. If the test succeeds, the events are causally related. Otherwise they are independent.



\end{document}

\documentclass[11pt,a4paper]{article}
\usepackage[T1]{fontenc}
\usepackage{lmodern}
\usepackage[latin1]{inputenc}
\usepackage{ngerman}
\usepackage{a4wide}
\usepackage[dvips]{graphicx}
\usepackage{listings}

\usepackage{ace}

\usepackage[
pdfauthor={ ACE Projekt Team },
pdftitle={ Projekthandbuch },
pdfcreator={pdftex},
]{hyperref}

\begin{document}

\tableofcontents

\section*{Versionskontrolle}

\begin{table}[!h]
 \begin{tabular}{|l|l|l|l|}
  \hline
  Version     & Datum       & Verantwortlich  &  Bemerkungen \\
  \hline
  0.1         & 15.03.2005  & rasss           &  Erste Version \\
  \hline
 \end{tabular}
 \caption{Versionskontrolle}
 \label{Versionskontrolle}
\end{table}

\begin{table}[!h]
 \begin{tabular}{|l|l|l|l|l|}
  \hline
                         & Stelle      & Datum & Visum & Bemerkungen \\
  \hline
  \textbf{Freigegeben}   & Projektteam &       &       &             \\
  \hline
  \textbf{Genehmigt}     &             &       &       &             \\
  \hline
 \end{tabular}
 \caption{Pr�fung/Genehmigung}
 \label{Pr�fung/Genehmigung}
\end{table}

\section{Einleitung}

\subsection{Zweck des Dokuments}

Das Projekthandbuch dient als einheitliche Handlungsgrundlage f�r alle Projektbeteiligten und legt damit den allgemeing�ltigen technischen und organisatorischen Rahmen fest. 

Dieses Dokument ist soweit wie m�glich als statisches Dokument zu f�hren. Es ist jedoch zu Beginn und am Schluss jeder Phase zu �berpr�fen und an die neuen Erkentnisse anzupassen.


\section{Projektbeschreibung}

Das gemeinsame Editieren von Dokumenten kann eine grosse Herausforderung sein. Versionierungs Systeme wie Subversion und CVS k�nnen einer Gruppe helfen, ein Dokument gemeinsam zu bearbeiten. Diese Systeme erm�glichen aber nicht ein gemeinsames Bearbeiten in Echtzeit. Genau an dieser Stelle setzt das Projekt \emph{ACE} an. Es soll ein Texteditor implementiert werden, der das gleichzeitige Bearbeiten des gleichen Dokumentes in Echtzeit erm�glicht.

\subsection{Ausgangssituation}
Heutzutage besitzt praktisch jeder Mensch in den Industriel�ndern einen Computer oder hat einen Zugang dazu. Verfassen von Texten, e-Mails und das Verwenden des World Wide Web geh�ren f�r die meisten Menschen zum Alltag. Doch gerade im Bereich Zusammenarbeit erleichtern uns Computer die Arbeit oft nicht. Im Gegenteil, oft verhindern sie ein intuitives Zusammenarbeiten. Allgemein ist man der �berzeugung, dass Anwendungen, welche die Zusammenarbeit f�rdern und erleichtern, ein grosses Potential haben. Insbesondere in einer Zeit wo SMS, Instant Messaging, Mobil- und Internet-Telefonie zum Alltag geh�ren.

\subsection{Ziele}
In der Semesterarbeit soll die Basis gelegt werden f�r die Implementation eines kollaborativen Texteditors im Rahmen der Diplomarbeit.

\begin{table}[!h]
 \begin{tabular}{|l|l|}
 \hline
 Ziel  & Priorit�ten & Beschreibung \\
 \hline
 1     & 1 & Aufbau von Knowhow im Bereich CSCW \\
 \hline
 2     & 1 & Bestehende Algorithmen evaluieren (Vor-/Nachteile) \\
 \hline
 3     & 1 & Algorithmus implementieren (Synchronisationsengine) \\
 \hline
 4     & 1 & Testframework f�r Algorithmus (Testf�lle) \\
 \hline
 5     & 2 & Konzept GUI \\
 \hline
 6     & 2 & Konzept Netzwerk/Kommunikation \\
 \hline
 7     & 3 & User Stories f�r kollaborativen Texteditor \\
 \hline
 \end{tabular}
 \caption{Ziele}
 \label{Ziele}
\end{table}

\subsection{Vorgehensstrategie}



\section{Projektspezifisches Vorgehensmodell}




\section{Entscheidungspunkte und auszuliefernde Ergebnisse}




\section{Methoden und Werkzeuge}

\subsection{Dokumente}
Zum Erstellen aller Dokumente soll grunds�ztlich \LaTeX{} verwendet werden. Selbstverst�ndlich werden f�r Grafiken und Diagramme andere Anwendungen verwendet. Da es sich bei \LaTeX{} Dateien um Textdateien handelt, k�nnen wir SubEthaEdit (ein kollaborativer Editor f�r Mac) verwenden, um zur gleichen Zeit am gleichen Dokument zu arbeiten.

\subsection{Source Repository}
Subversion wird als Source Repository verwendet. Die URL f�r den Subversion Zugriff ist http://ace.iserver.ch:81/repos/ace/ace.

\subsection{Projektwebsite}
Von der Projektwebsite aus erreicht man alle wichtigen Werkzeuge. Die Projektwebsite ist erreichbar unter http://ace.iserver.ch. Die Projektwebsite dient auch dazu, das Produkt \emph{ACE} zu vermarkten. 

\subsection{Zeiterfassung}
Zur Erfassung der Arbeitszeit wird eine PHP Applikation, entwickelt von Herrn Bigler, verwendet. Dies wird es erm�glichen, am Ende des Projektes einige Angaben zu den geleisteten Arbeitsstunden zu machen.

\subsection{Trac}
Trac ist ein sogenannter issue tracker mit einem integrierten Wiki. Das Wiki wird f�r den projektinternen Informationsaustausch verwendet. Der issue tracker wird f�r die gezielte Erfassung von Bug Reports verwendet.

\subsection{Kalender}
Ein online verf�gbarer Projektkalender gibt Auskunft �ber alle anstehenden Sitzungen und andere Termine.

\subsection{Entwicklung}
\begin{itemize}
 \item Eclipse (http://www.eclipse.org/)
 \item SubEthaEdit (http://www.codingmonkeys.de/subethaedit/)
 \item ant (http://ant.apache.org/)
\end{itemize}


\section{Standards und Richtlinien}

\subsection{Dokumentation}
Alle \LaTeX{} Dokumente sollen die Package ace.tex verwenden. Diese Datei findet man im Subversion Repository unter \texttt{/ace/trunk/doc/latex/ace.tex}. Eine Vorlage f�r alle \LaTeX{} Dokumente findet man im selben Verzeichnis (template.tex).

\subsection{Quellcode}
Die Dokumentation des Quellcodes erfolgt mit JavaDoc. Die Kommentare werden einheitlich in Englisch verfasst. Quellcodedateien werden mit folgendem Header, welcher unter \texttt{/ace/trunk/doc/templates/source.header} zu finden ist, versehen.

\lstset{language=Java}
\small{
\begin{verbatim}
/*
 * ace - a collaborative editor
 * Copyright (C) 2005 Mark Bigler, Simon Raess, Lukas Zbinden
 *
 * This program is free software; you can redistribute it and/or
 * modify it under the terms of the GNU General Public License
 * as published by the Free Software Foundation; either version 2
 * of the License, or (at your option) any later version.
 *
 * This program is distributed in the hope that it will be useful,
 * but WITHOUT ANY WARRANTY; without even the implied warranty of
 * MERCHANTABILITY or FITNESS FOR A PARTICULAR PURPOSE.  See the
 * GNU General Public License for more details.
 *
 * You should have received a copy of the GNU General Public License
 * along with this program; if not, write to the Free Software
 * Foundation, Inc., 59 Temple Place - Suite 330, Boston, MA  02111-1307, USA.
 */
\end{verbatim}
}

\subsubsection{Konventionen}
Die Quellcodedateien sollen gem�ss den Sun Coding Conventions (http://java.sun.com/docs/codeconv/) erstellt werden. Die Einr�ckung erfolgt mit Tabulator Zeichen (nicht mit Spaces).


\appendix
\section{ Appendix }
code fragments


\listoftables
\listoffigures

\end{document}

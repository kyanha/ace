\documentclass[11pt,a4paper]{article}
\usepackage[T1]{fontenc}
\usepackage{lmodern}
\usepackage[latin1]{inputenc}
\usepackage{ngerman}
\usepackage{a4wide}
\usepackage[dvips]{graphicx}
\usepackage{listings}

\usepackage{ace}

\usepackage[
pdfauthor={ ACE Projekt Team },
pdftitle={ Projekthandbuch },
pdfcreator={pdftex},
]{hyperref}

\begin{document}

\section{Zweck des Dokuments}

\section{Projektbeschreibung}

\subsection{Ausgangssituation}

\subsection{Ziele}

\subsection{Vorgehensstrategie}


\section{Projektspezifisches Vorgehensmodell}


\section{Entscheidungspunkte und auszuliefernde Ergebnisse}


\section{Methoden und Werkzeuge}

\subsection{Dokumente}
Zum Erstellen aller Dokumente soll grunds�ztlich \latex verwendet werden. Selbstverst�ndlich werden f�r Grafiken und Diagramme andere Anwendungen verwendet. Da es sich bei \latex Dateien um Textdateien handelt, k�nnen wir SubEthaEdit (ein kollaborativer Editor f�r Mac) verwenden, um zur gleichen Zeit am gleichen Dokument zu arbeiten.

\subsection{Source Repository}
Subversion wird als Source Repository verwendet. Die URL f�r den Subversion Zugriff ist http://ace.iserver.ch:81/repos/ace/ace.

\subsection{Projektwebsite}
Von der Projektwebsite aus erreicht man alle wichtigen Werkzeuge. Die Projektwebsite ist erreichbar unter http://ace.iserver.ch. Die Projektwebsite dient auch dazu, das Produkt \emph{ACE} zu vermarkten. 

\subsection{Zeiterfassung}
Zur Erfassung der Arbeitszeit wird eine PHP Applikation, entwickelt von Herrn Bigler, verwendet. Dies wird es erm�glichen, am Ende des Projektes einige Angaben zu den geleisteten Arbeitsstunden zu machen.

\subsection{Trac}
Trac ist ein sogenannter issue tracker mit einem integrierten Wiki. Das Wiki wird f�r den projektinternen Informationsaustausch verwendet. Der issue tracker wird f�r die gezielte Erfassung von Bug Reports verwendet.

\subsection{Kalender}
Ein online verf�gbarer Projektkalender gibt Auskunft �ber alle anstehenden Sitzungen und andere Termine.

\subsection{Entwicklung}
\begin{itemize}
 \item Eclipse (http://www.eclipse.org/)
 \item SubEthaEdit (http://www.codingmonkeys.de/subethaedit/)
 \item ant (http://ant.apache.org/)
\end{itemize}


\section{Standards und Richtlinien}

\subsection{Dokumentation}
Alle \latex Dokumente sollen die Package ace.tex verwenden. Diese Datei findet man im Subversion Repository unter \texttt{/ace/trunk/doc/latex/ace.tex}. Eine Vorlage f�r alle \latex Dokumente findet man im selben Verzeichnis (template.tex).

\subsection{Quellcode}
Die Dokumentation des Quellcodes erfolgt mit JavaDoc. Die Kommentare werden einheitlich in Englisch verfasst. Quellcodedateien werden mit folgendem Header, welcher unter \texttt{/ace/trunk/doc/templates/source.header} zu finden ist, versehen.

\lstset{language=Java}
\begin{lstlisting}
/*
 * ace - a collaborative editor
 * Copyright (C) 2005 Mark Bigler, Simon Raess, Lukas Zbinden
 *
 * This program is free software; you can redistribute it and/or
 * modify it under the terms of the GNU General Public License
 * as published by the Free Software Foundation; either version 2
 * of the License, or (at your option) any later version.
 *
 * This program is distributed in the hope that it will be useful,
 * but WITHOUT ANY WARRANTY; without even the implied warranty of
 * MERCHANTABILITY or FITNESS FOR A PARTICULAR PURPOSE.  See the
 * GNU General Public License for more details.
 *
 * You should have received a copy of the GNU General Public License
 * along with this program; if not, write to the Free Software
 * Foundation, Inc., 59 Temple Place - Suite 330, Boston, MA  02111-1307, USA.
 */
\end{lstlisting}

\subsubsection{Konventionen}
Die Quellcodedateien sollen gem�ss den Sun Coding Conventions (http://java.sun.com/docs/codeconv/) erstellt werden. Die Einr�ckung erfolgt mit Tabulator Zeichen (nicht mit Spaces).


\appendix

\section{Anhang: Erg�nzende Projektvereinbarungen}

\subsection{Projektorganisation}

\subsection{Projektplanung}

\subsection{Qualit�tssicherung}

\subsection{Konfigurationsmanagement}

\subsection{Projektrandbedingungen}

\subsection{Sicherheit}

\end{document}

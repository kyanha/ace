\chapter{Decisions for Network Implementation}
\label{chapter:decisionsnetwork}

In the semester project, we evaluated possible solutions to use for the network implementation (see report \emph{Report Evaluation Network}). Here we only justify the decision and do not repeat the potential solutions.

\section{Requirements}
We had to choose between technologies for the following requirements:

\begin{itemize}
\item dynamic service discovery - discover running services of the same type in a network
\item communication framework - a framework to support the implementation of an application protocol
\item platform and progamming language independence
\item open source and freely available
\end{itemize}


\section{Decision criterias}
In order to select a particular network technology or a combination of several technologies we had to answer some basic questions. 

\subsubsection{Open Architecture} 
Do we want to have an open or a closed protocol. If we decide to have an open, public protocol, other applications could collaborate with our application. We would have to define the protocol and make this information publicly available. An open protocol would greatly benefit if it were not tied to any programming language, that is in our case Java. Jini/RMI for instance would pretty much limit the collaboration with other applications to Java applications. A network layer based on either JXTA or a combination of Bonjour and BEEP would be programming language independent and thus more broadly applicable. 

The tendency was clear to have an open protocol since the project is going to be open source. Further, the solution should be platform and programming language independent to make ACE accessible to as many developers as possible in the future.

\section{Service discovery}
For the service discovery the decision fell on Bonjour zero-conf networking technology. It is based on open standards and has implementations for several programming languages including Java. An important argument was that Bonjour has been used successfully by many companies especially for printer discovery but also for many other kinds of discovery. It has a small API and is easy to use. 

The decision did not fall on Jini mainly because it is programming language dependent. JGroups did not match with our requirements since it is a pure peer to peer (P2P) solution and hence not suitable for the client-server \texttt{Jupiter} algorithm.

See section \ref{chapter:frameworks.bonjour} for an introduction on  Bonjour.

\section{Communcation}
The decision for the communication solution was harder because \emph{RMI}, \emph{JXTA} or \emph{BEEP} were all acceptable solutions. The only drawback about RMI was that it is language dependent. With JXTA this constraint is futile. So we considered on that solution. But as tests showed, the current version of JXTA is not thoroughly stable. This seemed a high risk for choosing that solution. Finally we checked BEEP. The evaluation for that technology were convincing. BEEP is platform and language independent and also available for Java. BEEP also matched our requirement to have a framework for application protocol implementation the most.

Thus, we decided to use the BEEP framework for the implementation of the ACE protocol. See section \ref{chapter:framework.beepcore} for an introduction on BEEP.


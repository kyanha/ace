\chapter{Overview}
\label{chapter:overview}



\section{What is Collaborative Editing?}



\section{Use Cases of Collaborative Editing}



\section{Requirements of a Collaborative Application}
A synchronous collaborative editor has some quite unique requirements.
The following requirements have been identified for such an editor
by Sun et al. (\cite{sun98a}).

\subsection{Real-time} 
The response to local user actions must be quick, ideally
as quick as with a single-user editor. Imposing a global total order on
message transmission is not
an option because of the distributed nature of the system. Thus, generally 
these systems replicate the document at each user's site.

\subsection{Distributed} 
Cooperating users reside on different machines 
connected by communication networks with nondeterministic latency.

\subsection{Unconstrained} 
Multiple users are allowed to concurrently and
independently edit any part of the document at any time, in order to 
facilitate free and natural information flow among multiple users.



\section{Previous Work}
\label{sect:overview.previouswork}

The diploma project is based on the work done in the semester project. Some
additional work has been done from from the end of the semester
project to the start of the diploma project. In this section we show which
results are not part of the diploma project, but have been implemented
before.

\subsection{Semester Project}
In the semester project we have implemented the following features:
\begin{itemize}
 \item consistency control algorithm (\emph{Jupiter})
 \item test framework for algorithm
\end{itemize}

In the first stage of the semester project we evaluated different
consistency control algorithm. The \emph{Report Evaluation Algorithm}
documents the evaluation process. The implementation of the
algorithm is documented in \emph{Report Implementation Algorithm}. The
test framework is documented in \emph{Report Testframework}.

Futher, we did some research to find out how we want to implement the
network as well as the application layer. The results of this research
can be found in the \emph{Report Evaluation Network} and 
\emph{Report Evaluation GUI} respectively.


\subsection{Summer Time}
During the summer break we tried to finish the implementation of undo/redo
in the algorithm. However, we stumbled across some issues that could not
be resolved in reasonable time. Read section \ref{sect:algorithm.undoredo} for
a detailed description of the problem.

We also investigated how difficult it would be to create a collaborative 
editing plugin for jEdit (see \href{http://jedit.org/}{http://jedit.org/}).
More about this investigation can be read in chapter 
\ref{chapter:decisionsapplication}.


\subsection{NICE}
\label{algo:nice}

Haifeng Shen and Chengzhen Sun devised a new operational transformation control algorithm in combination with a notification component in 2002 \cite{sun02}. In this paper they describe a flexible notification framework that can be used to implement a wide range of notification strategies used in collaborative systems.

In the proposed framework, the notification policy that determines when and what to notify is separated from the notification mechanism that determines how to notify. The parameters \emph{frequency} and \emph{granularity} are provided to define various notification policies. 

The frequency parameter determines the \emph{when} aspect of notification, that is, when a notification is propagated/accepted. The granularity parameter determines the \emph{what} aspect of notification, that is, which updates are going to be propagated/accepted. Further there can be a separate policy for input and output direction.

The notification mechanism determines the \emph{how} aspect. Both outgoing and incoming messages are put in distinct buffers (input buffer IB and output buffer OB). An outgoing notification executor (ONE) and an incoming notification executor (INE) are needed to carry out various outgoing and incoming notification policies respectively.

A very important component in the notification mechanism is the notification propagation protocol (NPP), which is needed for propagating updates from the OB at the notifying site to the IB at the notified site.

The notifications are contextually serialized. This is achieved by use of a central notification server, which acts as a centralized serialization point and message relaying agent (all messages pass through this central server). 


\subsubsection{Notification Propagation Protocol}
Before propagating a notification, the notifying site sends a \emph{Token-Request} message to the \emph{Notifier} (a central notification server), waiting for the \emph{Token-Grant} message from the \emph{Notifier}. After being granted the token, the site propagates the notification piggybacked with the \emph{Token-Release} message to the \emph{Notifier}. When the \emph{Notifier} receives the notification and the \emph{Token-Release} message, it forwards the notification to all interested sites. By using the notifier as a message relaying agent, causal relationships among notifications are automatically guaranteed.

This sequential propagation simplifies concurrency control. However it is also inefficient in supporting notification policies for meeting real-time collaboration needs. For propagating one notification that may contain only one operation, three extra messages have to be sent.


\subsubsection{Concurrent Propagation}
The proposed solution consists of a protocol that allows a site to propagate its notification without first requesting a token, thus effectively eliminating the \emph{Token-Request} message. This operation propagation protocol is called \emph{SCOP} (symmetric contextually-serialized operation propagation). 


\subsubsection{Properties}
The transformation control algorithm is called \emph{SLOT} (symmetric linear operation transformation). Together with the \emph{SCOP} protocol it has the following properties.

\begin{itemize}
 \item no state vectors needed
 \item no ET
 \item free of TP2
 \item architecture: replicated, unicast
\end{itemize}

The reason why \emph{SLOT} is free of TP2 is that under no circumstance an operation could be transformed against the same pair of operations in different orders. The operations are always ordered uniquely.

